\documentclass[t,usepdftitle=false]{beamer}

\usepackage[utf8]{inputenc}
\usetheme{Warsaw}
\usepackage{xcolor}

\usepackage{etex}
\usepackage{pictex}
\usepackage{tikz}
\usetikzlibrary{shapes,arrows}
\usepackage{pgfplots}

\usepackage{amsmath}
\usepackage{amscd}
\usepackage{pstricks, pst-node}

\usepackage{times}

\usepackage{ulem}

\usepackage{amsmath, amsthm}
\usepackage{listings}

% \setbeamercovered{transparent}
%\usecolortheme{crane}
\title[IFT3245]{IFT 3245\\Simulation et modèles}
\author[Fabian Bastin]{Fabian Bastin\\DIRO\\Université de Montréal}
\date{Automne 2016}

\def\be{\boldsymbol{e}}
\def\bh{\boldsymbol{h}}
\def\bu{\boldsymbol{u}}
\def\bv{\boldsymbol{v}}
\def\bx{\boldsymbol{x}}
\def\bA{\boldsymbol{A}}
\def\bP{\boldsymbol{P}}
\def\bU{\boldsymbol{U}}
\def\bX{\boldsymbol{X}}
\def\bY{\boldsymbol{Y}}
\def\bmu{\boldsymbol{\mu}}
\def\bSigma{\boldsymbol{\Sigma}}
\def\bzero{\boldsymbol{0}}

\def\eqas{\overset{\mbox{p.s.}}{=}}

\def\cH{\mathcal{H}}
\def\cJ{\mathcal{J}}
\def\cS{\mathcal{S}}

\def\rit{\mathcal{R}}
\def\NN{\mathcal{N}}
\def\RR{\mathcal{R}}
\def\ZZ{\mathcal{Z}}

\def\eps {$< 10^{-15}$}
\def\epsm {$> 1-10^{-15}$}

\def\MSE{\mbox{MSE}}
\def\RE{\mbox{RE}}
\def\eff{\mbox{Eff}}
\def\Var{\mbox{Var}}
\def\Cov{\mbox{Cov}}
\def\var{\mbox{var}}

\def\iid{i.i.d.}
\def\toas{\mbox{ to as }}
\def\To{\overset{D}{\to}}

\newtheorem{thm}{Theorème}
\newtheorem{defn}{Définition}
\newtheorem{coro}{Corollaire}
\newtheorem{prop}{Proposition}
\newtheorem{exe}{Exemple}

\begin{document}
\frame{\titlepage}

\begin{frame}
\frametitle{Comparaison de deux systèmes similaires}

Supposons à présent qu'il est possible de diminuer légèrement le paramètre $\gamma$ à de la loi gouvernant les durées de service, en prenant
\[
  \gamma = {\gamma_2} = \gamma_1(1-\delta),
\]
pour ${\delta} > 0$ très petit.

\mbox{}

Si on utilise les m\^emes nombres aléatoires, cela équivaut \`a multiplier
les durées de service par $(1-\delta)$.

\mbox{}

Nous voulons estimer $\mu(\gamma_2) - \mu(\gamma_1) = E_\gamma[X_2] - E_\gamma[X_1]$,
afin de mesurer l'effet d'augmenter un peu la vitesse des serveurs.

\end{frame}

\begin{frame}
\frametitle{Comparaison de deux systèmes similaires}

On simule $n$ jours pour chaque valeur de $\gamma$.
\begin{itemize}
\item
${X_{1,i}}=$ valeur de $G_i(s)$ avec $\gamma_1$;
\item
${X_{2,i}}=$ valeur de $G_i(s)$ avec $\gamma_2$;
\item
${\Delta_i} = X_{2,i}-X_{1,i}$,
\[
  {\bar\Delta_n} = {1\over n} \sum_{i=1}^n \Delta_i
\]
\end{itemize}

\mbox{}

On peut simuler $X_{1,i}$ et $X_{2,i}$
\begin{enumerate}[(i)]
\item
avec des v.a.{} indépendantes (\emph{VAI}),
\item
avec des v.a.{} communes (\emph{VAC}).
\end{enumerate}

\end{frame}

\begin{frame}
\frametitle{Comparaison de deux systèmes similaires}

Comme
\[
  \Var[\Delta_i] = \Var[X_{1,i}] + \Var[X_{2,i}] - 2\Cov[X_{1,i},\,X_{2,i}],
\]
le but des variables aléatoires communes est de rendre la covariance positive.

\mbox{}

Comment implanter les VAC?

\mbox{}

Utiliser des ``random streams'' différents pour générer:
\begin{enumerate}
\item
le facteur d'achalandage $B_i$;
\item
les temps inter-arrivées;
\item
les durées des appels;
\item
les durées de patience.
\end{enumerate}
Tout est généré par inversion:  on utilise une uniforme par v.a.

\end{frame}

\begin{frame}
\frametitle{Synchronisation}

Lorsqu'on change les durées de service, les durées d'attente changent
et les décisions d'abandon peuvent ainsi changer.

\mbox{}

Si on ne génère les durées de service que pour les clients qui 
n'abandonnent pas, alors on peut perdre la synchronisation:
on peut avoir une durée de service de moins \`a générer dans 
un système que dans l'autre.

\mbox{}

On peut générer les durées de service:
%
\begin{itemize}
\item[(a)] 
  pour tous les appels, m\^eme les abandons,
\item[(b)] 
  seulement pour les appels servis.
\end{itemize}

\end{frame}

\begin{frame}
\frametitle{Synchronisation}

De m\^eme, la durée de patience n'a pas besoin d'\^etre générée pour 
les clients qui n'attendent pas.  On peut la générer:
\begin{itemize}
\item[(c)] 
  pour tous les appels,
\item[(d)] 
  seulement si nécessaire.
\end{itemize}

\end{frame}

\begin{frame}
\frametitle{Synchronisation: résultats (avec ${n=10^4}$)}

\begin{footnotesize}
\begin{center}
\begin{tabular}{|l|rr|rr|rr|}
\hline
% &&&&&&\\
  \ Method
 & \multicolumn{2}{|c}{$\delta = 0.1$}
 & \multicolumn{2}{|c}{$\delta = 0.01$}
 & \multicolumn{2}{|c|}{$\delta = 0.001$} \\
 & $\bar \Delta_n$ & $\widehat{\Var}[\Delta_i]$
 & $\bar \Delta_n$ & $\widehat{\Var}[\Delta_i]$
 & $\bar \Delta_n$ & $\widehat{\Var}[\Delta_i]$\\
\hline
IRN (a + c)                      &  55.2 & 56913 &  4.98 & 45164 &  0.66 & 44046 \\
IRN (a + d)                      &  52.2 & 54696 &  7.22 & 45192 & -1.82 & 45022 \\
IRN (b + c)                      &  50.3 & 56919 &  9.98 & 44241 &  1.50 & 45383 \\
IRN (b + d)                      &  53.7 & 55222 &  5.82 & 44659 &  1.36 & 44493 \\
CRN, no sync. (b + d)            &  56.0 &  3187 &  5.90 &  1204 &  0.19 &   726 \\
CRN (a + c)                      &  56.4 &  2154 &  6.29 &    37 &  0.62 &   1.8 \\
CRN (a + d)                      &  55.9 &  2161 &  6.08 &   158 &  0.74 &  53.8 \\
CRN (b + c)                      &  55.8 &  2333 &  6.25 &   104 &  0.63 &   7.9 \\
CRN (b + d)                      &  55.5 &  2323 &  6.44 &   143 &  0.59 &  35.3 \\
% CRN (a + c), str.\ (opt.) & &&&&& \\ 
% CRN (a + d), str.\ (opt.) & &&&&& \\
% CRN (b + c), str.\ (opt.) & &&&&& \\
% CRN (b + d), str.\ (opt.) & &&&&& \\
\hline
\end{tabular}
\end{center}
\end{footnotesize}

\end{frame}

\begin{frame}
\frametitle{Induction de corrélation}

Conditions suffisantes pour que $\Cov[X,Y]$ 
soit positive, ou soit négative?

\mbox{}

Comment maximiser ou minimiser la corrélation pour des lois 
 marginales données?

\mbox{}

\begin{thm}[Bornes de Fréchet]
Parmi les paires de v.a.{} $(X,Y)$ dont les f.r.{} marginales sont 
${F}$ et ${G}$, la paire $(X,Y) = (F^{-1}(U), G^{-1}(U))$ 
o\`u ${U} \sim U(0,1)$, 
\emph{maximise} $\rho[X, Y]$, \
et la paire $(X,Y) = (F^{-1}(U), G^{-1}(1-U))$
\emph{minimise} $\rho[X, Y]$.
\end{thm}

\end{frame}

\begin{frame}
\frametitle{Induction de corrélation}

Par exemple, si la fonction de répartition d'une durée de service est 
${F}$ dans le premier système et ${G}$ dans le second,
et si on génère les durées de service par
$X = F^{-1}(U)$ et $Y = G^{-1}(U)$, alors $\Cov[X, Y] \ge 0$.

\begin{thm}
Soient \ ${X} = {f}(U)$ o\`u ${U} \sim U(0,1)$ et \
 ${Y} = {g}(V)$ o\`u ${V} \sim U(0,1)$. Alors,
\begin{itemize}
\item
si $f$ et $g$ sont monotones dans le m\^eme sens et $V=U$,
alors $\Cov[X, Y] \ge 0$;
\item
si $f$ et $g$ sont monotones dans le m\^eme sens et $V=1-U$,
alors $\Cov[X, Y] \le 0$.
\end{itemize}
\end{thm}

\end{frame}

\begin{frame}
\frametitle{Valeurs aléatoires communes (VAC)}

On utilise ${\Delta} = {X_2}-{X_1}$ pour estimer 
${\mu_2}-{\mu_1} = E[X_2] - E[X_1]$.  On a
\[
 \Var[\Delta] = \Var[X_1] + \Var[X_2] - 2 \, \Cov[X_1,X_2].
\]
\emph{Objectif}: induire une corrélation positive entre $X_1$ et $X_2$
sans changer leurs lois individuelles.

\mbox{}

Technique: utiliser les m\^emes nombres aléatoires pour simuler
les deux systèmes, en essayant de maintenir la \emph{synchronisation}
le mieux possible.

\mbox{}

Si $X_k = {f_k}(\bU_k) = f_k({U_{k,1}},{U_{k,2}},\dots)$ 
pour $k=1,2$, utiliser des VAC partout veut dire prendre $\bU_1=\bU_2$.

\end{frame}

\begin{frame}
\frametitle{Valeurs aléatoires communes (VAC)}

\begin{thm}
Si $f_1$ et $f_2$ sont monotones dans le m\^eme sens par rapport \`a
tous leurs arguments, alors  $\Cov[X_1, X_2] \ge 0$.
\end{thm}

\mbox{}

Pour \emph{maximiser} la corrélation, il faudrait générer
$X_1$ et $X_2$ directement par inversion!

\end{frame}

\begin{frame}
\frametitle{Exemple: Processus de Lindley}

${W_{i+1}} = \max(0,\, W_i + S_i - A_i)$, o\`u $S_i-A_i$ est
indépendant de $W_i$.

\mbox{}

Si ${X_1}$ et ${X_2}$ sont des fonctions non-décroissantes des $W_i$
pour deux processus de Lindley simulés avec des VAC, 
alors $\Cov[X_1, X_2] \ge 0$.

\mbox{}

Parfois il est très difficile de vérifier les conditions du
théorème.

\mbox{}

Par exemple, pour la banque ou un centre d'appels, la monotonicité
est difficile \`a vérifier \`a cause des possibilités d'abandon.
De plus, les conditions du théorème sont \emph{suffisantes}, mais 
\emph{pas nécessaires}.

\end{frame}

\begin{frame}
\frametitle{Valeurs aléatoires communes (VAC)}

Pour \emph{tester} si c'est efficace en pratique:
faire une \emph{expérience pilote avec les VAC} et estimer 
$\Cov(X_1, X_2)$.

\mbox{}

Il n'est pas nécessaire de faire de simulations sans les VAC:
pour estimer la variance qu'on aurait sans les VAC,
il suffit de prendre la version empirique de $\Var[X_1] + \Var[X_2]$.

\end{frame}

\end{document}