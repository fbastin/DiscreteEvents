\documentclass[t,usepdftitle=false]{beamer}
\usepackage[utf8]{inputenc}
\usetheme{Warsaw}
\usepackage{xcolor}

\usepackage{etex}
\usepackage{pictex}
\usepackage{tikz}
\usetikzlibrary{shapes,arrows}
\usepackage{pgfplots}

\usepackage{amsmath}
\usepackage{amscd}
\usepackage{pstricks, pst-node}

\usepackage{times}

\usepackage{ulem}

\usepackage{amsmath, amsthm}
\usepackage{listings}

% \setbeamercovered{transparent}
%\usecolortheme{crane}
\title[IFT3245]{IFT 3245\\Simulation et modèles}
\author[Fabian Bastin]{Fabian Bastin\\DIRO\\Université de Montréal}
\date{Automne 2016}

\def\be{\boldsymbol{e}}
\def\bh{\boldsymbol{h}}
\def\bu{\boldsymbol{u}}
\def\bv{\boldsymbol{v}}
\def\bx{\boldsymbol{x}}
\def\bA{\boldsymbol{A}}
\def\bP{\boldsymbol{P}}
\def\bX{\boldsymbol{X}}
\def\bY{\boldsymbol{Y}}

\def\cH{\mathcal{H}}
\def\cJ{\mathcal{J}}
\def\cS{\mathcal{S}}

\def\rit{\mathcal{R}}
\def\NN{\mathcal{N}}
\def\RR{\mathcal{R}}
\def\ZZ{\mathcal{Z}}

\def\eps {$< 10^{-15}$}
\def\epsm {$> 1-10^{-15}$}

\def\MSE{\mbox{MSE}}
\def\RE{\mbox{RE}}
\def\eff{\mbox{Eff}}
\def\Var{\mbox{Var}}
\def\var{\mbox{var}}

\def\iid{i.i.d.}
\def\toas{\mbox{ to as }}
\def\To{\overset{D}{\to}}

\begin{document}
\frame{\titlepage}

\begin{frame}
\frametitle{Estimation séquentielle}

Pour un intervalle de confiance de niveau $1-{\alpha}$, si on fixe ${n}$, la largeur 
$I_2-I_1$ est aléatoire.

\mbox{}

Si on veut $I_2 - I_1 \le {w}$ pour $w$ fixé, la valeur minimale de $n$
requise est une variable aléatoire ${N}$.

\mbox{}

Comment \emph{prédire} ce $N$? 

\mbox{}

Pour $X_i \sim$ \emph{binomiale}$(1,{p})$, avec $n=1000$ on a obtenu 
$\overline{X}_n = 0.882$,  $S_n^2 \approx {0.1042}$, et la demi-largeur du
intervalle de confiance à 95\%{} était de 0.020.
Combien de répétitions additionnelles faut-il pour réduire
la demi-largeur à environ {0.005}?

\end{frame}

\begin{frame}
\frametitle{Estimation séquentielle}

Nous voulons $1.96 S_n/\sqrt{n} \le 0.005$.
En supposant que $S_n$ ne changera pas significativement, cela donne
$n \ge (1.96\times S_n / 0.005)^2 \approx 16011.8$.
En conséquence, nous pouvons recommander de faire 15012 répétitions
additionnelles.

\end{frame}

\begin{frame}
\frametitle{Procédure à deux étapes}

Cette approche est valable pour la loi de Student (ou normale).

\mbox{}

Faire ${n_0}$ répétitions et calculer $S^2_{n_0}$;
la prédiction du $n$ requis est 
\[
  {\hat{N}^*} 
  = \min\left\{n \mid (t_{n-1,1-\alpha/2}) S_{n_0}/\sqrt{n}\le r\right\}.
\]
On fera $\max(0,\ \hat{N}^*-n_0)$ répétitions additionnelles.

\mbox{}

Bien sûr, il se peut que ce soit insuffisant, ou trop.

\mbox{}

Après $n_0$, recalculer $S_n^2$ et la demi-largeur pour chaque $n$.
On s'arrête dès que $I_2 - I_1 \le w$.

\end{frame}

\begin{frame}
\frametitle{Procédure à deux étapes}

Cette procédure est biaisée, car on tend à s'arrêter à un
$N$ où $S_N^2$ \emph{sous-estime} la variance.  

\mbox{}

Mais lorsque $w\to 0$, le bias disparaît, 
$N/n^* \to 1$ a.p.1
où ${n^*}$ est la valeur optimale de $N$ si on connaissait $\sigma^2$,
et $P[|\overline{X}_N - \mu| \le w/2] \to 1-\alpha$.

\end{frame}

\begin{frame}
\frametitle{IC pour des estimateurs de quantile}

Si ${X}$ est de répartition ${F}$, le
\emph{$q$-quantile} de $F$ est 
\[
  {\xi_q} = F^{-1}(q) = \inf\{x : F(x) \ge {q}\}.
\]
Soit $X_{(1)},\dots,X_{(n)}$ un échantillon \iid{} de $X$, trié,
et $\hat F_n$ la fonction de répartition empirique.
Un estimateur simple de $\xi_q$ est le \emph{quantile empirique}
\[
  {\hat\xi_{q,n}} = \hat F_n^{-1}(q) = \inf\{x : \hat F_n(x) \ge q\}
            = X_{(\lceil nq\rceil)}.
\]
Il est biaisé mais fortement consistent et obéit au théorème de la
limite centrale, comme le montre le théorème ci-après.

\end{frame}

\begin{frame}
\frametitle{IC pour des estimateurs de quantile: théorème}

\textsl{
\begin{enumerate}[(i)]
\item
Pour chaque $q$, $\hat\xi_{q,n} \toas \xi_q$ quand $n\to\infty$.
\item
Si $X$ a une densité ${f}$ strictement positive et continue 
 dans un voisinage de $\xi_q$, alors
\[
  \frac{\sqrt{n} (\hat\xi_{q,n} - \xi_q) f(\xi_q)}{ \sqrt{q(1-q)}}
    \To N(0,1)  \quad\mbox{ quand } n\to\infty.
\]
\end{enumerate}
}

\mbox{}

Ce TLC indique qu'il y a beaucoup de bruit (variance) si $f(\xi_q)$
est petit.

\mbox{}

De plus, pour l'utiliser afin de construire un intervalle de
confiance, il faut estimer $f(\xi_q)$, ce qui est difficile.

\end{frame}

\begin{frame}
\frametitle{IC pour des estimateurs de quantile}

Nous pouvons néanmoins construire une méthode non-asymptotique de
calcul d'un intervalle de confiance pour $\xi_q$: supposons que $F$ est continue en $\xi_q$.

\mbox{}

Soit ${B}$ le nombre d'observations $X_{(i)}$ inférieures à $\xi_q$.

\mbox{}

Puisque $P[X < \xi_q] = q$, $B$ est binomiale$(n,q)$.

\mbox{}

Si $1\le j < k \le n$, $X_{(j)} < \xi_q \leq X_{(k)}$ ssi $j \leq B < k$.
Alors
\[
 P[{X_{(j)}} < \xi_q \le {X_{(k)}}]
 =  P[j\le B < k] 
 = \sum_{i=j}^{k-1} {n\choose i}  q^i(1-q)^{n-i}. 
\]
On choisit ${j}$ et ${k}$ pour que cette somme soit supérieure ou
égale à $1-\alpha$ (intervalle unilatéral ou bilatéral).

\end{frame}

\begin{frame}
\frametitle{IC pour des estimateurs de quantile}

Si $n$ est grand et $q$ n'est pas trop proche de 0 ou 1, on peut
approximer la loi binomiale par la loi \emph{normale}:
\[
\frac{B - nq}{\sqrt{nq(1-q)}} \approx N(0,1).
\]
On obtient alors $j = \lfloor nq + 1 - \delta\rfloor$ et 
$k = \lfloor nq + 1 + \delta\rfloor$, où
$\delta = \sqrt{nq(1-q)}\Phi^{-1}(1-\alpha/2)$.

\end{frame}

\begin{frame}
\frametitle{Exemple: valeur à risque}

Soit ${L}$ la perte nette de valeur d'un porte-feuille d'actifs pour
une période de temps donnée $[0,{T}]$.
La valeur à risque (VAR) (au
temps 0) est la valeur de ${x_p}$ telle que $P[L > x_p] = p$.
C'est le $(1-p)$-quantile de $L$.

\mbox{}

Valeurs courantes: $p = 0.01$, $T=$ 2 semaines (banques),
$T=$ mois ou années (assurance, fonds de pension).

\mbox{}

On peut critiquer l'utilisation de la VAR, vu qu'elle
donne une information très limitée.

\mbox{}

Par exemple si $x_{0.01} = 10^7$ dollars, que sait-on sur l'importance
réelle de la perte?

\mbox{}

Une mesure complémentaire pourrait être $E[L \mid L > x_p]$, 
par exemple.

\end{frame}

\begin{frame}
\frametitle{Exemple: valeur à risque}

Modèles pour estimer la VAR: on doit modéliser l'évolution du prix des
actifs (souvent plusieurs milliers, dépendants). Souvent: modèles à
facteurs.

\mbox{}

On peut remplacer les actifs par des prêts, comptes à payer, etc.
Sauf dans les cas simples, on estime la VAR par simulation.
Quand $p$ est petit: ``importance sampling''. 

\end{frame}

\end{document}
