\documentclass[t,usepdftitle=false]{beamer}
\usepackage[utf8]{inputenc}
\usetheme{Warsaw}
\usepackage{xcolor}
% \setbeamercovered{transparent}
%\usecolortheme{crane}
\title[IFT3245]{IFT 3245\\Simulation et modèles}
\author[Fabian Bastin]{Fabian Bastin\\DIRO\\Université de Montréal}
\date{Automne 2016}

\begin{document}
\frame{\titlepage}

% ------------------------------------------------------------------------------------------------------------------------------------------------------
\begin{frame}
\frametitle{Objectifs}

\begin{enumerate}
\item
Comprendre les enjeux principaux des modèles de simulations informatiques.
\item
Distinguer les divers types de modèles, et identifier les situations nécessitant le recours à la simulation.
\item
Notions d'événements discrets.
\item
Etre capable de construire les éléments d'une simulation à événements discrets.
\item
Collecter les résultats d'une simulation et les interpréter.
\item
Améliorer autant que possible la qualité de ces résultats.
\item
Illustration des concepts au moyen de modèles de transport.
\end{enumerate}

\end{frame}

\begin{frame}
\frametitle{Support de cours}

\begin{itemize}
\item
Notes de cours disponibles sur Studium (sujette à révision)
\item
Livre de référence conseillé (non obligatoire): 
A. M. Law, Simulation Modeling and Analysis, quatrième édition, McGraw-Hill, USA, 2007.
\end{itemize}

\end{frame}

\begin{frame}
\frametitle{Concepts de base}

\begin{description}
\item[Système]
collection d'entités qui agissent et interagissent afin d'accomplir une certaine fin logique.\\
L'état d'un système est la collection de variables nécessaires pour décrire un système à un instant particulier.
\item[Modèle]
description simplifiée d'un systeme, dans le but d'évaluer sa performance ou l'effet de certaines décisions.
\item[Simulation]
faire évoluer le modèle d'un système en fournissant les entrées appropriées, puis observer et analyser les résultats.
\end{description}

\mbox{}

Deux types principaux de systèmes: discrets et continus.

\mbox{}

Pourquoi simuler?

\end{frame}

\begin{frame}
\frametitle{Modèles}

Modèle physique vs modèle mathématique.

\mbox{}

Modèle analytique, numérique, par simulation.

\end{frame}

\begin{frame}
\frametitle{Logiciels}

Programmation sous Julia: \url{http://www.julialang.org}

\mbox{}

Paquetages Julia: SimJulia, Distributions, RandomStreams.

\mbox{}

Motivations:
\begin{itemize}
\item
développer et comprendre les principes informatiques de la simulation par événéments discrets, en ``grattant'' le code;
\item
un prototype de simulation de trafic sera ajouté par la suite.
\end{itemize}

\end{frame}

\begin{frame}
\frametitle{Pourquoi Julia?}

Langage relativement nouveau, mais une base d'utilisateurs en croissance exponentielle, et de nombreux développeurs.

\mbox{}

Avantages: open-source, multi-plateforme, langage simple, peut être interfacé avec les langages traditionnels: C, C++, Fortran, Python,\ldots

\mbox{}

Peut être interprété ou compilé.

\mbox{}

Inconvénients: certains fonctions sont appelées à évoluer, et certaines librairies doivent être améliorées.

\end{frame}

\begin{frame}
\frametitle{Évaluation du cours}

4 devoirs, comptant pour 50\% de la note finale. Chacune des trois premiers devoirs vaudra 10\%. Le dernier 20\%.

\mbox{}

Examen intra: 20\%

\mbox{}

Examen fina: 30\%

\end{frame}

\end{document}

