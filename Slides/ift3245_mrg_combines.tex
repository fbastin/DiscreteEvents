\documentclass[t,usepdftitle=false]{beamer}
\usepackage[utf8]{inputenc}
\usetheme{Warsaw}
\usepackage{xcolor}

\usepackage{etex}
\usepackage{pictex}
\usepackage{tikz}
\usetikzlibrary{shapes,arrows}
\usepackage{pgfplots}

\usepackage{amsmath}
\usepackage{amscd}
\usepackage{pstricks, pst-node}

\usepackage{times}

\usepackage{ulem}

\usepackage{amsmath, amsthm}
\usepackage{listings}

% \setbeamercovered{transparent}
%\usecolortheme{crane}
\title[IFT3245]{IFT 3245\\Simulation et modèles}
\author[Fabian Bastin]{Fabian Bastin\\DIRO\\Université de Montréal}
\date{Automne 2016}

\def\be{\boldsymbol{e}}
\def\bh{\boldsymbol{h}}
\def\bu{\boldsymbol{u}}
\def\bv{\boldsymbol{v}}
\def\bx{\boldsymbol{x}}
\def\bA{\boldsymbol{A}}

\def\cJ{\mathcal{J}}
\def\cS{\mathcal{S}}

\def\rit{\mathcal{R}}
\def\RR{\mathcal{R}}
\def\ZZ{\mathcal{Z}}

\begin{document}
\frame{\titlepage}


\begin{frame}
\frametitle{MRGs combinés}

Considérons deux [ou plusieurs...] MRGs évoluant en parallèle:
\begin{eqnarray*}
 x_{1,n} &=& (a_{1,1} x_{1,n-1} + \cdots + a_{1,k} x_{1,n-k}) \mod m_1,\\
 x_{2,n} &=& (a_{2,1} x_{2,n-1} + \cdots + a_{2,k} x_{2,n-k}) \mod m_2.
\end{eqnarray*}

\mbox{}

On définit les deux \emph{combinaisons}:
$$
\begin {array}{rclrcl}
    z_n &:=& (x_{1,n} - x_{2,n}) \mod m_1; &
    {u_n} &:=& z_n/m_1; \\
    {w_n} &:=& (x_{1,n}/m_1 - x_{2,n}/m_2) \mod 1.
\end {array}
$$

\mbox{}

La suite ${\{w_n,\, n\ge 0\}}$ est la sortie d'un autre MRG,
de module ${m} = m_1 m_2$, et
${\{u_n,\, n\ge 0\}}$ est presque la même suite si $m_1$ et $m_2$
sont proches.  Peut atteindre la période $(m_1^k-1)(m_2^k-1)/2$.

\mbox{}

Permet d'implanter efficacement un MRG ayant un grand $m$ et
plusieurs grands coefficients non nuls.
% Critéres: \\
%  (1) les composantes s'implantent efficacement; \\
%  (2) la combinaison a une bonne uniformité.\\

\end{frame}

\begin{frame}
\frametitle{MRGs combinés}

Pour accélérer la génération de point, il est possible de prendre tous
les $a_j$ non nuls égaux à $a$  (Deng et Xu 2002).
Alors, $x_n = a (x_{n-i_1} + \cdots + x_{n-k}) \mod m$.
Une seule  multiplication.

\mbox{}

Les meilleurs générateurs ne jouissent cependant pas de cette propriété.

Tableaux de paramétres: L'Ecuyer (1999);
L'Ecuyer et Touzin (2000).

\end{frame}

\begin{frame}
\frametitle{MRG32K3a}

$J=2$, $k=3$, \\
$m_1 = 2^{32} -209$, $a_{11} = 0$, $a_{12} = 1403580$, $a_{13} = -810728$,\\
$m_2 = 2^{32}-22853$, $a_{21} = 527612$, $a_{22} = 0$, $a_{23} = -1370589$.\\
Combination: $z_n = (x_{1,n} - x_{2,n}) \bmod m_1$.

\mbox{}

Le générateur correspond à un MRG caractérisé par $k=3$, $m = m_1 m_2
= 18446645023178547541$, et les paramétres $a_{1} =
18169668471252892557$, $a_{2} = 3186860506199273833$, $a_{3} =
8738613264398222622$. Sa période $\rho$ vaut $(m_1^3-1)(m_2^3-1)/2
\approx 2^{191}$.

\end{frame}

\begin{frame}[fragile]
\frametitle{MRG32K3a}

\begin{footnotesize}
\begin{verbatim}
 #define norm 2.328306549295728e-10  /* 1/(m1+1) */
 #define m1   4294967087.0
 #define m2   4294944443.0
 #define a12     1403580.0
 #define a13n     810728.0
 #define a21      527612.0
 #define a23n    1370589.0

 double  s10, s11, s12, s20, s21, s22;

 double MRG32k3a ()
 {
    long   k;
    double p1, p2;
\end{verbatim}
\end{footnotesize}


\end{frame}

\begin{frame}[fragile]
\frametitle{MRG32K3a}

\begin{footnotesize}
\begin{verbatim}

    /* Component 1 */
    p1 = a12 * s11 - a13n * s10;
    k = p1 / m1;   p1 -= k * m1;   if (p1 < 0.0) p1 += m1;
    s10 = s11;   s11 = s12;   s12 = p1;

    /* Component 2 */
    p2 = a21 * s22 - a23n * s20;
    k  = p2 / m2;  p2 -= k * m2;   if (p2 < 0.0) p2 += m2;
    s20 = s21;   s21 = s22;   s22 = p2;

    /* Combination */
    if (p1 <= p2) return ((p1 - p2 + m1) * norm);
    else return ((p1 - p2) * norm);
 }
\end{verbatim}
\end{footnotesize}

\end{frame}

\end{document}
