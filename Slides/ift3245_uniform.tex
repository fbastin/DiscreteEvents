\documentclass[t,usepdftitle=false]{beamer}
\usepackage[utf8]{inputenc}
\usetheme{Warsaw}
\usepackage{xcolor}

\usepackage{etex}
\usepackage{pictex}
\usepackage{tikz}
\usetikzlibrary{shapes,arrows}
\usepackage{pgfplots}

\usepackage{amsmath}
\usepackage{amscd}
\usepackage{pstricks, pst-node}

\usepackage{times}

\usepackage{ulem}

\usepackage{amsmath, amsthm}
\usepackage{listings}

% \setbeamercovered{transparent}
%\usecolortheme{crane}
\title[IFT3245]{IFT 3245\\Simulation et modèles}
\author[Fabian Bastin]{Fabian Bastin\\DIRO\\Université de Montréal}
\date{Automne 2016}

\def\bu{\boldsymbol{u}}
\def\bx{\boldsymbol{x}}
\def\cS{\mathcal{S}}

\begin{document}
\frame{\titlepage}

% ------------------------------------------------------------------------------------------------------------------------------------------------------
\begin{frame}
\frametitle{Générateur $U(0,1)$: principe de base}

Définir une fonction de transition \[
f: \mathcal{S} \rightarrow \mathcal{S},\]
où $\mathcal{S}$ est l'espace d'état, de cardinalité
finie.

\mbox{}

L'état initial: $s_0$. Récurrence:
\[
s_n = f(s_{n-1}).
\]
Supposons de plus que $f$ est périodique pour tout $n \geq \tau$ connu (souvent égal à 0), de période $\rho \leq \# \cS$.

\mbox{}

Espace de sortie: $\mathcal{U} = (0,1)$.

\mbox{}

Fonction de sortie
\[
g: \mathcal{S} \rightarrow \mathcal{U}
\]
transforme l'état $s_n$ dans la valeur de sortie $u_n$.

\end{frame}

\begin{frame}

\[
\begin{CD}
 \cdots @>f>> s_{\rho-1} @>f>> s_0 @>f>> s_1  @>f>> 
  \cdots @>f>> s_n @>f>> \cdots \\ 
%
  @. @V{g}VV  @V{g}VV   @V{g}VV   
  @.  @V{g}VV  \\
%
 \cdots @. u_{\rho-1} @. u_0
 @.  u_1  @.   \cdots @.  u_n @.  \cdots \\
\end{CD}
\]
\label{fig:rng}

\mbox{}

{\bf Comment choisir $f$ et $g$?}

\mbox{}

Buts: grand $\rho$, bonne uniformité, comportement "aléatoire".

\end{frame}

\begin{frame}
\frametitle{Générateur congruentiel linéaire (GCL)}

Dès 1948 furent introduits des générateurs de la forme
\[
ax + c\mod m.
\]

En supposant tout d'abord que $c$ vaut 0 (comme dans l'approche proposée par Lehmer), la période maximale est $m-1$ et est atteinte si et seulement si $m$ est premier et $a$ est
une racine primitive de $m$.

\mbox{}

$r$ est une racine primitive de $m$ si les puissances de $r$ ($1$, $r$,
$r^2$, $r^3$,\ldots) génèrent tous les entiers non-nuls modulo $m$.

\end{frame}

\begin{frame}
\frametitle{Générateur congruentiel linéaire (GCL)}

Puisqu'il y a $m-1$ entiers non nuls, ceci signifie que les premières $m-1$ puissances de $r$ doivent être différentes, modulo $m$.

\mbox{}

De manière équivalente, nous pouvons parler de l'ordre de $r$.
L'ordre d'une racine $r$ de $m$ est le plus petit entier (strictement)
positif $x$ tel que $r^x = 1 \mod m$.

\mbox{}

$r$ est une racine primitive si son ordre est $m-1$.
Il est possible de montrer que ceci équivaut à exiger que $a^{(m -
  1)/p} - 1$ est un multiple de $m$ pour chaque facteur premier $p$ de
$m - 1$, ou encore le plus petit entier $l$ pour lequel $r^l-1$ est
divisible par $m$ est $l = m-1$. % (Knuth~\cite{Knut88}, page 20).

\end{frame}

\begin{frame}
\frametitle{Générateur congruentiel linéaire (GCL)}

Les générateurs congruentiels linéaires qui remplissent ces conditions
sont appelés GCL's multiplicatifs à modulus premier.

\mbox{}

Notons que la condition $m$ premier suffit pour garantir l'existence d'un générateur de période maximale,
en vertu du théorème ci-dessous.

\mbox{}

{\red Théorème}\\
\textit{
Si $m$ est premier, il existe une racine primitive pour $m$.
}

\mbox{}

Il n'existe malheureusement pas de méthode simple pour calculer ces racines.\\

\end{frame}

\begin{frame}
\frametitle{GCL: exemple}

%\begin{exe}
Si $m=7$, alors 3 est une racine primitive de $m$ car les puissances
de 3 modulo 7 sont 1, 3, 2, 6, 4, 5, c'est-à-dire chaque entier strictement
compris entre 0 et 7.
Mais 2 n'est pas une racine primitive de $m$ car les puissances de $2$ modulo 7
sont 1, 2, 4, 1, 2, 4, 1, 2, 4,\ldots
%\end{exe}

\end{frame}

\begin{frame}
\frametitle{Générateur congruentiel linéaire (GCL)}

Si $c \ne 0$, il est possible d'obtenir une période égale à $m$, sous
les conditions exposées dans le théoréme ci-dessous.
\textit{
Le GCL a une période pleine si et seulement si les trois conditions suivantes tiennent:
\begin{enumerate}
\item
le seul entier positif qui divise de manière exacte à la fois $m$ et
$c$ est 1;
\item
si $q$ est un nombre premier qui divise $m$, alors $q$ divise $a-1$;
\item
si 4 divise $m$, alors $4$ divise $a-1$.
\end{enumerate}
}

\end{frame}

\begin{frame}
\frametitle{Générateur standard minimal}

Exemple.\\
%\begin{exe}
Park et Miller ont proposé un générateur standard qu'ils ont appelé le Standard
Minimal générateur Standard Minimal, aprés avoir testé divers générateurs connus au moment de leur étude.

\mbox{}

Bien qu'il suffise pour les applications simples,
les générateurs présentés dans les sections suivantes le surpassent
largement, et par conséquent, il est déconseillé de l'utiliser pour
des simulations complexes.

\mbox{}

Le Standard Minimal est un générateur congruentiel linéaire défini par
la récurrence
\[
  x_{n+1} = 16807x_n\mod (2^{31}- 1).
\]
\end{frame}

\begin{frame}
\frametitle{Implantation de générateurs congruentiels linéaires}

Une difficulté principale est de calculer $ax \mod m$ pour de grands $m$, ce qui entraîne des risques de débordement de registres.

\mbox{}

Première approche. Factorisation approximative.

\mbox{}

Cette méthode est valide si
\[
a^2 < m
\]
ou
\[
a = \lfloor m/i \rfloor,
\]
avec $i^2 < m$, et procède par des calculs sur des entiers.
\end{frame}

\begin{frame}
\frametitle{Factorisation approximative}

Précalculons ${q} := \lfloor m/a \rfloor$ et ${r} := m\mod a$, puis
\begin{align*}
  y &:= \lfloor x / q \rfloor; \\
  x &:= a (x - yq) - yr.
\end{align*}
Si $x < 0$, nous posons $x := x + m$. Justification:
\begin{eqnarray*}
  a x \mod m &=&  (ax - \lfloor x/q\rfloor m) \mod m \\
      &=&  (ax - \lfloor x/q\rfloor (a q + r)) \mod m \\
      &=&  (a (x - \lfloor x/q\rfloor q) - \lfloor x/q\rfloor r) \mod m \\
      &=&  (a (x \mod q) - \lfloor x/q\rfloor r) \mod m.
\end{eqnarray*}
Sous les conditions posées, il est immédiat de noter que toutes les quantités intermédiaires demeurent entre $-m$ et $m$.
\end{frame}

\begin{frame}[fragile]
\frametitle{Factorisation approximative: implantation}

En C, la procédure peut s'exprimer comme suit:
\begin{footnotesize}
\begin{verbatim}
  long q, r, y;

  q = m/a;
  r = m%a;

  y = x/q;
  x = a*(x-y*q)-y*r;

  if (x < 0) x += m;
\end{verbatim}
\end{footnotesize}
\end{frame}

\begin{frame}[fragile]
\frametitle{Calculs en point flottant, double précision.}

La procédure est valide si tous les entiers à considérer peuvent être représentés de manière exacte en passant en calcul flottant.
En particulier, si la double précision fait appel à 64 bits, et suit
la norme IEEE, la procédure suivante est correcte si $am < 2^{53}$:
\begin {eqnarray*}
   && \mbox{double } m, a, x, y; \quad  \mbox{int } k; \\
   && y = a * x; \quad
    k = \lfloor y / m \rfloor; \quad
    x = y - k * m;
\end {eqnarray*}
\end{frame}

\begin{frame}
\frametitle{Décomposition en puissances de 2.}

Supposons que ${a = \pm 2^{q} \pm 2^{r}}$ et ${m = 2^{e} - {h}}$ pour
$h$ petit.
Dans ce cas,
\[
ax\mod m = \pm 2^q x \mod m + \pm 2^rx \mod m.
\]
Pour calculer $y = 2^q x$ mod $m$ (le calcul de $2^r x$ est
similaire), nous décomposons $x$ en ${x_0} + 2^{e-q}{x_1}$.

\mbox{}

\setlength{\unitlength}{0.7cm}
\begin{picture}(5,3.2)(0,0)
\put(3,1){\line(1,0){8}}
\put(3,2){\line(1,0){8}}
\put(3,1){\line(0,1){1}}
\put(11,1){\line(0,1){1}}
\put(8,1){\line(0,1){1}}
%\put(12,1.5){\large $e$-bit number$\mod m$ }
\put(1.5,1.3){\large $x=$}
\put(5,1.3){\large $x_1$}
\put(8.8,1.3){\large $x_0$}
\put(5,2.3){$q$ bits}
\put(8.8,2.3){$(e-q)$ bits}
\end{picture}
\end{frame}

\begin{frame}
\frametitle{Décomposition en puissances de 2.}

Pour $h=1$ (Wu, 1997), on obtient $y$ en permutant $x_0$ et $x_1$.
En effet,
\begin{align*}
2^q x\mod m &= 2^q(x_0 + 2^{e-q}x_1) \mod (2^e - 1) \\
&= 2^q x_0 + [2^e x_1 \mod (2^e-1)] \\
&= 2^q x_0 + x_1.
\end{align*}
\end{frame}

\begin{frame}
\frametitle{Décomposition en puissances de 2.}

Pour $h>1$ (L'Ecuyer et Simard 1999), nous avons de la même manière
\begin{eqnarray*}
  y = 2^q(x_0 + 2^{e-q}x_1) \mod (2^e - h) 
    = (2^q x_0 + hx_1) \mod (2^e - h).
\end{eqnarray*}%
Si $h < 2^q$ et $h (2^q -(h+1)2^{-e+q}) < m$, comme $x_0 \leq 2^{e-q}$
et $x_1 \leq 2^q$, nous avons
\[
2^q x_0 \leq 2^e - 2^q < m.
\]
De plus, étant donné que $2^{e-q}x_1 \leq m-1$, nous avons
\[
hx_1 \leq h(m-1)/2^{e-q} = h(2^e-h-1)/2^{e-q} = h(2^q-(h+1)2^{-e+q}) < m,
\]
et par conséquent chaque terme est strictement inférieur à $m$.
L'opération modulo revient dès lors à soustraire $m$ si la somme est $\ge m$.
\end{frame}

\begin{frame}[fragile]
\frametitle{Décomposition en puissances de 2: implantation}

\begin{footnotesize}
\begin{verbatim}
#define m     1073741789    /* 2^30 - 35 */
#define h     35
#define q     15
#define emq   15            /* e - q */
#define mask1 32767         /* 2^(e-q) - 1 */
#define r     13
#define emr   17            /* e - r */
#define mask2 131071        /* 2^(e-r) - 1 */
#define norm  1.0/m

long x;
\end{verbatim}
\end{footnotesize}

\end{frame}

\begin{frame}[fragile]
\frametitle{Décomposition en puissances de 2: implantation}

\begin{footnotesize}
\begin{verbatim}
double axmodm () {
  unsigned long k, x0, x1;

  x0 = x & mask1;
  x1 = x >> emq;
  k = (x0 << q) + h*x1;

  x0 = x & mask2;
  x1 = x >> emr;
  k += (x0 << r) + h*x1;

  if (k < m) x = k;
  else if (k < 2*m) x = k-m;
  else x = k - 2*m;

  return x*norm;
}
\end{verbatim}
\end{footnotesize}

\end{frame}

\begin{frame}
\frametitle{Décomposition en puissances de 2.}

L'Ecuyer et Simard ont toutefois démontré que ces générateurs
présentent des faiblesses statistiques s'ils sont utilisés de manière
directe.

\end{frame}

\end{document}
