\documentclass[t,usepdftitle=false]{beamer}
\usepackage[utf8]{inputenc}
\usetheme{Warsaw}
\usepackage{xcolor}

\usepackage{etex}
%\usepackage{pictex}
\usepackage{tikz}
\usetikzlibrary{shapes,arrows}
\usepackage{pgfplots}

\usepackage{amsmath}
\usepackage{amscd}
\usepackage{pstricks, pst-node}

\usepackage{times}

\usepackage{ulem}

\usepackage{amsmath, amsthm}
\usepackage{listings}

% \setbeamercovered{transparent}
%\usecolortheme{crane}
\title[IFT3245]{IFT 3245\\Simulation et modèles}
\author[Fabian Bastin]{Fabian Bastin\\DIRO\\Université de Montréal}
\date{Automne 2016}

\def\be{\boldsymbol{e}}
\def\bh{\boldsymbol{h}}
\def\bu{\boldsymbol{u}}
\def\bv{\boldsymbol{v}}
\def\bx{\boldsymbol{x}}
\def\bA{\boldsymbol{A}}

\def\cH{\mathcal{H}}
\def\cJ{\mathcal{J}}
\def\cS{\mathcal{S}}

\def\rit{\mathcal{R}}
\def\RR{\mathcal{R}}
\def\ZZ{\mathcal{Z}}

\def\eps {$< 10^{-15}$}
\def\epsm {$> 1-10^{-15}$}

\begin{document}
\frame{\titlepage}

\begin{frame}
\frametitle{Tests statistiques}

L'étude des propriétés théoriques d'un générateur ne suffit;
il est indispensable de recourir à des tests statistiques pour valider celui-ci.

\mbox{}

Nous formulons l'hypothèse
\[
 {\cH_0}\,:\, \mbox{``}\{u_0,u_1,u_2,\dots\} \mbox{ sont des variables aléatoires i.i.d. } U(0,1)\mbox{''}.
\]

\mbox{}

Nous savons que $\cH_0$ est fausse, mais comment pouvons-nous le détecter?

\mbox{}

Pour ce faire, définissons une statistique ${T}$, fonction de $u_i$, dont la distribution
sous $\cH_0$ est connue (ou approximativement).
Nous rejetons $\cH_0$ si la valeur de $T$ est trop extrême.

\end{frame}

\begin{frame}
\frametitle{Tests statistiques}

La puissance et l'efficacité du test dépendent fortement de la classe d'alternatives.
Différents tests détectent des problémes différents.

\mbox{}

L'idéal serait de disposer d'une statistique $T$ qui imite la variable aléatoire d'intérêt pratique.
Ce n'est cependant guére réalisable en pratique.

\mbox{}

Impossible de construire un générateur de variables aléatoires qui passerait tous les tests. Compromis: construire un générateur qui passe le plus de tests
raisonnables.

\mbox{}

La plupart des tests étudient la répartition uniforme d'un ensemble ou sous-ensemble quelconque de points produits par le générateur.

\end{frame}

\begin{frame}
\frametitle{Tests statistiques}

Avoir une bonne couverture uniforme ne suffit toutefois pas: les points générés doivent avoir l'apparence d'indépendance les uns par rapport aux autres.

\mbox{}

Les tests d'uniformité, qui représente la majeure partie des tests à notre disposition, remplissent partiellement ce second objectif dans la mesure où ils sont validés pour des sous-ensembles de points très variés.

\mbox{}

Certains tests cependant sont spécifiquement conçus pour tester les corrélations entre les points produits par un générateur.

\end{frame}

\begin{frame}
\frametitle{TestU01}

Le logiciel TestU01 (\url{http://www.iro.umontreal.ca/~simardr/testu01/tu01.html}), implémente un nombre très important de tels tests
Nous nous limiterons à quelques exemples afin d'illustrer les idées sous-jacentes.

\end{frame}

\begin{frame}
\frametitle{Test de collisions} 

Partitionnons l'hypercube $[0,1)^t$ en $k = d^t$ boîtes cubiques de
tailles égales.
Générons $n$ points $\bu_i = (u_{ti}, \dots, u_{ti+t-1})$ dans $[0,1)^t$, et
posons $X_j$ le nombre de points dans la boîte $j$.

\mbox{}

Le nombre de collisions est donné par
\[
  {C} = \sum_{j=0}^{k-1} \max(0,\, X_j-1).
\]

\mbox{}

Sous $\cH_0$, $C$ suit approximativement une loi de Poisson de moyenne $\lambda = n^2/(2k)$,
pour $n$ suffisamment grand, et un petit $\lambda$.

\end{frame}

\begin{frame}
\frametitle{Test de collisions} 

Si nous observons $c$ collisions, nous calculons la $p$-valeur à
droite comme
\[
   p^+(c) = P[X \ge c \mid X \sim \mbox{ Poisson}(\lambda)].
\]

\mbox{}

Nous rejetons $\cH_0$ si $p^+(c)$ est de manière consistante très
proche de 0 (trop de collisions) ou de 1 (pas assez de collisions).
%\  Here, we shall use $t=2$.

\mbox{}

Une interprétation possible est l'étude par simulation Monte Carlo du comportement probabiliste d'un algorithme de hachage.

\end{frame}

\begin{frame}
\frametitle{Espacement entre anniversaires}

Nous partitionnons à nouveau l'hypercube $[0,1)^t$ en $k=d^t$ boîtes
cubiques, et générons $n$ points.

\mbox{}

Soit $I_{(1)}\le I_{(2)}\le \cdots\le I_{(n)}$ les numéros de boîte où les points tombent, triés par ordre croissant.

\mbox{}

Nous calculons les espacements
\[
S_j = I_{(j+1)}-I_{(j)},\ 1\le j \le n-1.
\]

\mbox{}
Le nombre de collisions entre les espacements est défini comme
\[
  Y = \sum_{j=1}^{n-1} I\left[S_{(j+1)}=S_{(j)}\right].
\]

\end{frame}

\begin{frame}
\frametitle{Espacement entre anniversaires}

Pour $k$ suffisamment grand, $Y$ suit approximativement une loi de
Poisson de moyenne $\lambda = n^3/(4k)$.

\mbox{}

Si $Y$ prend la valeur $y$, la $p$-valeur à droite est
$$
   p^+(y) = P[X \ge y \mid X \sim \mbox{ Poisson}(\lambda)].
$$

\mbox{}

Interprétation comme problème de simulation: nous pouvons vouloir
étudier la distribution de $Y$ par simulation Monte Carlo, pour des valeurs modérées de $k$ et $n$.


\end{frame}

\begin{frame}
\frametitle{Test d'autocorrélation}

Générons $n$ uniformes $u_1,\ldots,u_n$ et calculons l'autocorrélation empirique de distance $k$:
\[
\hat{\rho}_k = \frac{1}{n-k} \sum_{j = 1}^{n-k} \left( u_ju_{j+k} - \frac{1}{4}\right).
\]

\mbox{}

La distribution empirique de $N$ valeurs de
\[\sqrt{12(n - k)}\hat{\rho}_k\]
est comparée avec la distribution normale standard,
qui est la distribution théorique asymptotique quand $n \rightarrow +\infty$.

\mbox{}

L'approximation n'est valide que pour $n$ très grand, et avec $k$ très
petit en comparaison de $n$.

\end{frame}

\begin{frame}
\frametitle{Quelques générateurs largement utilisés}

Avant 2003, le générateur dans MS Excel était
$$
   u_{i}  = (9821.0\,  u_{i-1} + 0.211327) \mod 1.
$$
Le générateur est à présent une combinaison de trois GCLs simplistes.
A la même période, le générateur dans MS VisualBasic était
\begin{eqnarray*}
  x_{i} &=& (1140671485\,  x_{i-1} + 12820163) \mod 2^{24},\\
  u_i   &=& x_i / 2^{24}. 
\end {eqnarray*}
Le générateur dans {\tt java.util.Random} de Java est quant à lui
\begin{eqnarray*}
   x_{i+1} &=& (25214903917\, x_i + 11) \mod 2^{48} \\
   u_i &=& [2^{27}\left(x_{2i} \mod 2^{26}\right) 
                    + x_{2i+1} \bmod 2^{27}] / 2^{53}
\end{eqnarray*}

\end{frame}

\begin{frame}
\frametitle{Résultats de tests: test de colisions}

Considérons $t = 2$, $d = n/16$, $\lambda = 128$.
\begin{center}
\begin {tabular}{|l|cc|cc|cc|}
\hline \rule{0pt}{11pt} $n$ & \multicolumn{2}{c|}{Java} &
 \multicolumn{2}{c|}{VisualBasic} & \multicolumn{2}{c|}{Excel} \\
 \rule{0pt}{11pt} & c & $p^+(c)$ & c & $p^+(c)$ & c & $p^+(c)$ \\
 \hline $2^{14}$ & & & & & 132 & \rule{0pt}{11pt}\\ 
 $2^{15}$ & & & 75 & $1-3.1\times 10^{-7}$ & 128 & \\ 
 $2^{16}$ & & & 38 & \epsm & 121 & \\
 $2^{17}$ & & &  0 & \epsm & 170 & $2.2\times 10^{-4}$\\ 
 $2^{18}$ & & &  0 & \epsm & 202 & $9.5\times 10^{-10}$\\ 
 $2^{19}$ & & &  0 & \epsm & 429 & \eps \\
%  $2^{20}$     &    &       & 0  &  \epsm & ---  &   --- \\
%  $2^{21}$     &    &       & 0  &  \epsm & ---  &   --- \\
%  $2^{22}$     & 156 &9.0\times 10^{-3}& 0  &  \epsm & ---  &   --- \\
%  $2^{23}$     &    &       & 0  &  \epsm & ---  &   --- \\
%  $2^{24}$     &    &       & 8388608  &  \eps      & ---  & --- \\
\hline
\end {tabular}
\end{center}

\end{frame}

\begin{frame}
\frametitle{Résultats de tests: test de colisions}

%\begin{comment}
Si on rejette les dix premiers bits de chaque $u_i$: on prend $u_i :=
1024 u_i \mod 1$.

\begin {table}[h]
\centering
\begin {tabular}{|l|cc|cc|cc|}
\hline
 \rule{0pt}{11pt} $n$  &  
\multicolumn{2}{c|}{Java} & 
\multicolumn{2}{c|}{VisualBasic} & 
\multicolumn{2}{c|}{Excel} \\
 \rule{0pt}{11pt} & c & $p^+(c)$ & c & $p^+(c)$ & c & $p^+(c)$ \\
\hline
  $2^{14}$     &    &       & 8192  & \eps   &   & \rule{0pt}{11pt}\\
  $2^{15}$     &    &       & 24576  &  \eps   &   & \\
  $2^{16}$     &    &       & 57344  &  \eps   &   & \\
  $2^{17}$     &    &       & 122880  &  \eps   &   & \\
  $2^{18}$     &    &       & 253952  &  \eps   & 224 & $1.0\times 10^{-14}$\\
  $2^{19}$     &160 & $3.5\times 10^{-3}$ & 516096  & \eps   & 425 & \eps \\
\hline
\end {tabular}
\end {table}

%\end{comment}

\end{frame}

\begin{frame}
\frametitle{Espacements d'anniversaire}

Considérons $t=2$, $\lambda = 1$.

\hspace{-1cm}
\begin{small}
\begin {tabular}{|ll|cc|cc|cc|}
\hline
 \rule{0pt}{11pt} $n$  &  $d$ &
\multicolumn{2}{c|}{Java} & 
\multicolumn{2}{c|}{VisualBasic} & 
\multicolumn{2}{c|}{Excel} \\
\rule{0pt}{11pt} && y & $p^+(y)$ & y & $p^+(y)$ & y & $p^+(y)$\\
\hline
 $2^{8}$  & $2^{11}$  &  & &    &  &  &  \rule{0pt}{13pt} \\
 $2^{10}$ & $2^{14}$  &  & & 10   & $1.1\times 10^{-7}$ &  & \\
 $2^{12}$ & $2^{17}$  &  & & 592  &\eps  & 5 & $3.7\times 10^{-3}$ \\
 $2^{14}$ & $2^{20}$  &  & & 11129   & \eps & 71 & \eps \\
 $2^{16}$ & $2^{23}$  &  & & 64063   & \eps & 558  & \eps \\
 $2^{18}$ & $2^{26}$ &14& $4.5\times 10^{-12}$ & 261604 & \eps & 4432 &\eps\\
\hline
\end {tabular}
\end{small}

\end{frame}

\begin{frame}
\frametitle{Espacements d'anniversaire}

\begin {center}
\begin {tabular}{|ll|cc|}
\hline
 \rule{0pt}{11pt} $n$  &  $d$ &
\multicolumn{2}{c|}{LCG-16807} \\
\rule{0pt}{11pt} && y & $p^+(y)$ \\
\hline
 $2^{10}$ & $2^{14}$  &  & \rule{0pt}{13pt} \\
 $2^{12}$ & $2^{17}$  &    2 & \\
 $2^{14}$ & $2^{20}$  &  170 & \eps \\
 $2^{16}$ & $2^{23}$  & 10060 & \eps \\
\hline
\end {tabular}
\end{center}

\end{frame}

\begin{frame}
\frametitle{Espacements d'anniversaire}

Et en rejetant les 10 premiers bits:
\begin {center}
\begin {tabular}{|l|cc|cc|cc|}
\hline
 \rule{0pt}{11pt} $n$  & 
\multicolumn{2}{c|}{Java} & 
\multicolumn{2}{c|}{VisualBasic} & 
\multicolumn{2}{c|}{Excel} \\
 \rule{0pt}{11pt} & y & $p^+(y)$ & y & $p^+(y)$ & y & $p^+(y)$ \\
\hline
  $2^{8}$    &   & &  52  &\eps   &  &  \rule{0pt}{13pt} \\
  $2^{9}$    &   & &  199  &\eps  &  &  \\
  $2^{10}$   &   & &  672   &\eps  &  & \\
  $2^{11}$   &   & &   1754  &  \eps & &  \\
  $2^{12}$   &   & &  3901  & \eps &  &  \\
  $2^{13}$   &   & &  8102  & \eps &  &  \\
  $2^{14}$   &   & &  16374  &\eps  &  &  \\
  $2^{15}$   & 21  &$6.0\times 10^{-15}$ & 32763   &\eps  &  &  \\
  $2^{16}$   & 99  &\eps & 65531   & \eps & 7 & $4.5\times 10^{-3}$  \\
  $2^{17}$   & 697  &\eps &  ---  &---  & 34 &   \eps \\
  $2^{18}$   & 639  & \eps  &---  &---  & 186 &   \eps  \\
\hline
\end {tabular}
\end {center}

\end{frame}

\begin{frame}
\frametitle{Tests systématiques pour des familles de générateurs de nombres aléatoires}

Pour une famille de générateurs de nombres aléatoires, on cherche une relation su type
$[
  n_0 \approx K \rho^{\gamma},
$]
où $n_0$ est la taille d'échantillon minimum pour obtenir un fort rejet,
$\rho$ est la longueur de période, et $K$ et $\gamma$ sont des constantes.

Pour des GCLs, on obtient (grossièrement) pour le test de collisions:
\[
n_0 \approx 16 \rho^{1/2}, 
\]
et pour le test d'espacement entre anniversaires:
\[
n_0 \approx 16 \rho^{1/3}. 
\]

Nous souhaitons dès lors construire le générateur de nombres aléatoires avec $\rho$ suffisamment grand de sorte que
générer $n_0$ nombres est infaisable en pratique.

\end{frame}

\begin{frame}
\frametitle{Librairie de tests} 

Tests effectués sur la plupart des générateurs existants.

\mbox{}

Pratiquement aucun des générateurs présent dans les logiciels
commerciaux ne passe tous les tests.
Vrai en particulier pour les générateurs modulo-2.

\mbox{}

Les vainqueurs sont des MRG avec une bonne période et une bonne structure,
des générateurs multiplicatifs ``lagged-Fibonacci'', des générateurs
non-linéaires conçus pour la cryptologie, et certains générateurs avec
des composantes venant de différentes familles.

\end{frame}

\begin{frame}
\frametitle{Librairie de tests} 

Il est possible de combiner efficacité empirique et théorique, comme l'illustre le MRG32k3a.

\mbox{}

Note: générateur multiplicatifs ``lagged-Fibonacci''. Forme de base
\[
x_n = x_{n-r}*x_{n-k} \mbox{ mod } m.
\]
Il a moins de propriétés mathématiques connues, difficultés du choix de $r$ et
de $k$, initialisation complexe.
$m$ est souvent pris comme une puissance de 2.

\end{frame}

\end{document}
