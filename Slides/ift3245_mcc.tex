\documentclass[t,usepdftitle=false]{beamer}

\usepackage[utf8]{inputenc}
\usetheme{Warsaw}
\usepackage{xcolor}

\usepackage{etex}
\usepackage{pictex}
\usepackage{tikz}
\usetikzlibrary{shapes,arrows}
\usepackage{pgfplots}

\usepackage{amsmath}
\usepackage{amscd}
\usepackage{pstricks, pst-node}

\usepackage{times}

\usepackage{ulem}

\usepackage{amsmath, amsthm}
\usepackage{listings}

% \setbeamercovered{transparent}
%\usecolortheme{crane}
\title[IFT3245]{IFT 3245\\Simulation et modèles\\
Esp\'erance conditionnelle (EC) \\ 
ou Monte Carlo conditionnel (CMC)}
\author[Fabian Bastin]{Fabian Bastin\\DIRO\\Université de Montréal}
\date{Automne 2016}

\def\be{\boldsymbol{e}}
\def\bh{\boldsymbol{h}}
\def\bu{\boldsymbol{u}}
\def\bv{\boldsymbol{v}}
\def\bx{\boldsymbol{x}}
\def\bA{\boldsymbol{A}}
\def\bC{\boldsymbol{C}}
\def\bD{\boldsymbol{D}}
\def\bI{\boldsymbol{I}}
\def\bM{\boldsymbol{M}}
\def\bP{\boldsymbol{P}}
\def\bU{\boldsymbol{U}}
\def\bX{\boldsymbol{X}}
\def\bY{\boldsymbol{Y}}

\def\bbeta{\boldsymbol{\beta}}
\def\bmu{\boldsymbol{\mu}}
\def\bnu{\boldsymbol{\nu}}
\def\bSigma{\boldsymbol{\Sigma}}
\def\bzero{\boldsymbol{0}}

\def\eqas{\overset{\mbox{p.s.}}{=}}

\def\cA{\mathcal{A}}
\def\cB{\mathcal{B}}
\def\cG{\mathcal{G}}
\def\cH{\mathcal{H}}
\def\cJ{\mathcal{J}}
\def\cL{\mathcal{L}}
\def\cS{\mathcal{S}}

\def\rit{\mathcal{R}}
\def\NN{\mathcal{N}}
\def\RR{\mathcal{R}}
\def\ZZ{\mathcal{Z}}

\def\eps {$< 10^{-15}$}
\def\epsm {$> 1-10^{-15}$}

\def\MSE{\mbox{MSE}}
\def\RE{\mbox{RE}}
\def\eff{\mbox{Eff}}
\def\Var{\mbox{Var}}
\def\Cov{\mbox{Cov}}
\def\var{\mbox{var}}

\def\names{}

\def\iid{i.i.d.}
\def\toas{\mbox{ to as }}
\def\To{\overset{D}{\to}}

\newtheorem{assumption}{Hypothèse}
\newtheorem{thm}{Theorème}
\newtheorem{defn}{Définition}
\newtheorem{coro}{Corollaire}
\newtheorem{prop}{Proposition}
\newtheorem{exe}{Exemple}

\def\Black{}
\def\yel{}
\def\tan{}
\def\rme{{\rm e}}

%\def\mid{\,|\,}

\def\eqdef{\overset{déf}{=}}

\begin{document}
\frame{\titlepage}

\begin{frame}
\frametitle{Un exemple simple}

Supposons que l'on veut estimer 
$\yel{\mu} = P[\yel{Y_1} + \cdots + \yel{Y_t} > \yel{x}]$.

Estimateur \'evident: $\yel{X} = I[Y_1+\cdots+Y_t > x]$.

Estimateur CMC: $\yel{X_{\rme,s}} = P[Y_1+\cdots+Y_t > x \mid Y_1,\dots,Y_s]$,
pour $\yel{s}\le t$.

\mbox{}

Si $s=t$: aucun changement.

Si $s=t-1$ et les $Y_j$ sont ind\'ependants:
$X_{\rme,t-1} = 1 - F_t[Y_1 + \cdots + Y_{t-1}]$.

Si $s=0$: $X_{\rme,0} = P[Y_1+\cdots+Y_t > x] = \mu$ \ 
(variance r\'eduite \`a z\'ero).

\mbox{}

En fait plus $s$ est petit, plus la variance est r\'eduite.

\end{frame}

\begin{frame}
	\frametitle{Cadre g\'en\'eral}
	
	\emph{Id\'ee}: remplacer l'estimateur $\yel{X}$ par $E[X\mid \yel{Z}]$
	o\`u $Z$ est une autre v.a., ou plus g\'en\'eralement par $E[X\mid \yel{\cG}]$,
	o\`u $\cG$ est une $\sigma$-alg\`ebre donnant une \emph{information partielle} sur $X$.
	
	\mbox{}

	L'estimateur CMC s'\'ecrit
	\[
	X_{\rme} \eqdef E[X\mid \cG]
	\]
	et on a
	\[
	E[X_{\rme}] = E[E[X\mid \cG]] = E[X].
	\]

\end{frame}

\begin{frame}
\frametitle{Cadre g\'en\'eral}

De plus,
	\[
	\Var[X] = E[{
		\underbrace{\Black\Var[X\mid \cG]}_{\begin{array}{c}
			\mbox{\names{\small Var.\ r\'esiduelle}}\\
			\mbox{\names{\small lorsque $\cG$ est connu}}\\ 
			\mbox{\names{\small (\'elimin\'e par CMC)}}\\
			\end{array}}}\hskip-10pt]  
	~+~ {\underbrace{\Black\Var[E[X\mid \cG]]}_{\begin{array}{c}
			\mbox{\names{\small Var.\ due \`a la}}\\  
			\mbox{\names{\small variation de $\cG$}}\\
			\end{array}}
	}
	\]
	et donc 
	\[
	\emph{\Var[X_{\rme}]} ~=~ \Var [X] - E[\Var[X\mid \cG]] ~\le~ \Var[X].
	\]
	(Cas particulier du th\'eor\`eme de Rao-Blackwell.)
	
\end{frame}

%%%%%%%%%%%%%%%%%%%%%%%%%%%%%%

\begin{frame}
	\frametitle{Cadre g\'en\'eral}
	
	Pour minimiser la variance, on doit maximiser $E[\Var[X\mid \cG]]$,
	i.e., \emph{$\cG$ doit contenir le moins d'information possible}.
	
	On peut en effet montrer que $\cG_1\subset\cG_2$ implique que 
	$E[\Var[X\mid \cG_1]] \ge E[\Var[X\mid \cG_2]]$.
	
	\emph{Mais} moins $\cG_1$ contient d'information, plus il est difficile
	de calculer $X_\rme$.
	
	On doit donc faire un \emph{compromis}.
	
	Dans certains cas, $X_\rme$ peut \^etre moins co\^uteux \`a calculer que $X$.
	
	\emph{Cas limites}:\\
	Si $\cG$ ne contient \emph{aucune information} pertinente \`a $X$:
	$\yel{X_\rme} = E[X\mid\cG] = E[X] = \mu$.\\
	Si $\cG$ permet de calculer $X$ (i.e., $X$ est \emph{$\cG$-mesurable}): 
	$\yel{X_{\rme}} = X$.
	
\end{frame}
%%%%%%%%%%%%%%%%%%%%%%%%%%%%%%

\begin{frame}[fragile]
	\frametitle{Exemple: r\'eseau d'activit\'es stochastique}
	
	On veut estimer $\mu = P[T > x]$.	
	Estimateur na\"{i}f: $\yel{X} = I[T > x]$.
	
	
	Soit $\yel{\cL}\subseteq\cA$ un ensemble d'activit\'es (arcs) tel que
	chaque chemin de la source au puits contient exactement un arc de $\cL$.
	($\cL$ est une coupe orient\'ee.) Exemple: $\cL = \{ V_4, V_5, V_6 \}$.

\begin{small}
	%
	\begin{center}
		%  Produced by xfig for pictex.
%  This file has been modified by hand.
%
\font\thinlinefont=cmr5
%
\begingroup\makeatletter\ifx\SetFigFont\undefined%
\gdef\SetFigFont#1#2#3#4#5{}%
%  \reset@font\fontsize{#1}{#2pt}%
%  \fontfamily{#3}\fontseries{#4}\fontshape{#5}%
%  \selectfont}% 
\fi\endgroup% 
%\mbox{
\beginpicture 
\setcoordinatesystem units <0.45cm,0.45cm> 
\unitlength=0.45cm 
\linethickness=2pt
\setplotsymbol ({\makebox(0,0)[l]{\tencirc\symbol{'160}}})
\setshadesymbol ({\thinlinefont .})  \setlinear % % Fig TEXT object %
\put{\SetFigFont{10}{12.0}{\rmdefault}{\bfdefault}{\updefault}\tan{puits}}
[lB] at 18.069 23.289 % % Fig TEXT object %
\put{\SetFigFont{10}{9.6}{\familydefault}{\mddefault}{\updefault}{$\names{V_5}$}}
[lB] at 9.366 18.415 % % Fig TEXT object %
\put{\SetFigFont{10}{9.6}{\rmdefault}{\mddefault}{\updefault}{$V_4$}}
[lB] at 9.049 16.192 % % Fig TEXT object %
\put{\SetFigFont{10}{9.6}{\rmdefault}{\mddefault}{\updefault} $V_3$}
[lB] at 6.985 19.209 % % Fig TEXT object %
\put{\SetFigFont{10}{9.6}{\rmdefault}{\mddefault}{\updefault}$\names{V_6}$} [lB]
at 10.160 22.860 % % Fig TEXT object %
\put{\SetFigFont{10}{9.6}{\rmdefault}{\mddefault}{\updefault}$V_7$} [lB]
at 12.700 16.192 % % Fig TEXT object %
\put{\SetFigFont{10}{9.6}{\rmdefault}{\mddefault}{\updefault}$V_8$} [lB]
at 12.065 18.098 % % Fig TEXT object %
\put{\SetFigFont{10}{9.6}{\rmdefault}{\mddefault}{\updefault}$\names{V_9}$} [lB]
at 13.652 19.526 % % Fig TEXT object %
\put{\SetFigFont{10}{9.6}{\rmdefault}{\mddefault}{\updefault} $\names{V_{10}}$}
[lB] at 13.018 21.590 % % Fig TEXT object %
\put{\SetFigFont{10}{9.6}{\rmdefault}{\mddefault}{\updefault}$V_{11}$}
[lB] at 14.605 23.019 % % Fig TEXT object %
\put{\SetFigFont{10}{9.6}{\rmdefault}{\mddefault}{\updefault}$\names{V_{12}}$}
[lB] at 15.875 17.939 % % Fig TEXT object %
\put{\SetFigFont{10}{9.6}{\rmdefault}{\mddefault}{\updefault}$V_{13}$}
[lB] at 16.828 21.590 % % Fig TEXT object %
\put{\SetFigFont{10}{9.6}{\rmdefault}{\mddefault}{\updefault}$V_1$} [lB]
at 4.921 17.145 % % Fig TEXT object %
\put{\SetFigFont{10}{9.6}{\rmdefault}{\mddefault}{\updefault}$V_2$} [lB]
at 5.239 20.161 % % Fig TEXT object %
\put{\SetFigFont{10}{12.0}{\rmdefault}{\bfdefault}{\updefault}\tan{source}}
[lB] at 1.587 17.621 % % Fig ELLIPSE % \linethickness= 0.500pt
\setplotsymbol ({\thinlinefont .})  
\iffalse  %%%%
\ellipticalarc axes ratio 0.502:0.502 360 degrees
        from  4.153 18.574 center at  3.651 18.574
%
% Fig ELLIPSE
%
\linethickness= 0.500pt
\setplotsymbol ({\thinlinefont .})
\ellipticalarc axes ratio  0.502:0.502  360 degrees 
        from 13.837 23.495 center at 13.335 23.495
%
% Fig ELLIPSE
%
\linethickness= 0.500pt
\setplotsymbol ({\thinlinefont .})
\ellipticalarc axes ratio  0.502:0.502  360 degrees 
        from 16.694 20.003 center at 16.192 20.003
%
% Fig ELLIPSE
%
\linethickness= 0.500pt
\setplotsymbol ({\thinlinefont .})
\ellipticalarc axes ratio  0.502:0.502  360 degrees 
        from 17.647 23.495 center at 17.145 23.495
%
% Fig ELLIPSE
%
\linethickness= 0.500pt
\setplotsymbol ({\thinlinefont .})
\ellipticalarc axes ratio  0.502:0.502  360 degrees 
        from 15.742 16.669 center at 15.240 16.669
%
% Fig ELLIPSE
%
\linethickness= 0.500pt
\setplotsymbol ({\thinlinefont .})
\ellipticalarc axes ratio  0.502:0.502  360 degrees 
        from  7.963 16.669 center at  7.461 16.669
%
% Fig ELLIPSE
%
\linethickness= 0.500pt
\setplotsymbol ({\thinlinefont .})
\ellipticalarc axes ratio  0.502:0.502  360 degrees 
        from 11.919 16.669 center at 11.417 16.669
%
% Fig ELLIPSE
%
\linethickness= 0.500pt
\setplotsymbol ({\thinlinefont .})
\ellipticalarc axes ratio  0.502:0.502  360 degrees 
        from  8.915 21.749 center at  8.414 21.749
%
% Fig ELLIPSE
%
\linethickness= 0.500pt
\setplotsymbol ({\thinlinefont .})
\ellipticalarc axes ratio  0.502:0.502  360 degrees 
        from 12.884 20.003 center at 12.383 20.003
\fi  %%%%%%%%%%%%%%
%
% Fig POLYLINE object
%
\linethickness= 0.500pt
\setplotsymbol ({\thinlinefont .})
\plot  8.890 22.066 12.859 23.336 /
%
% arrow head
%
\plot 12.636 23.198 12.859 23.336 12.597 23.319 /
%
%
% Fig POLYLINE object
%
\linethickness= 0.500pt
\setplotsymbol ({\thinlinefont .})
\plot  4.128 18.256  6.985 16.828 /
%
% arrow head
%
\plot  6.729 16.884  6.985 16.828  6.786 16.998 /
%
%
% Fig POLYLINE object
%
\linethickness= 0.500pt
\setplotsymbol ({\thinlinefont .})
\plot  7.620 17.145  8.414 21.273 /
%
% arrow head
%
\plot  8.428 21.011  8.414 21.273  8.303 21.035 /
%
%
% Fig POLYLINE object
%
\linethickness= 0.500pt
\setplotsymbol ({\thinlinefont .})
\plot 15.399 17.145 16.034 19.526 /
%
% arrow head
%
\plot 16.030 19.264 16.034 19.526 15.907 19.297 /
%
%
% Fig POLYLINE object
%
\linethickness= 0.500pt
\setplotsymbol ({\thinlinefont .})
\plot 16.351 20.479 16.986 23.019 /
%
% arrow head
%
\plot 16.986 22.757 16.986 23.019 16.863 22.788 /
%
%
% Fig POLYLINE object
%
\linethickness= 0.500pt
\setplotsymbol ({\thinlinefont .})
\plot 11.589 17.145 12.224 19.526 /
%
% arrow head
%
\plot 12.220 19.264 12.224 19.526 12.097 19.297 /
%
%
% Fig POLYLINE object
%
\linethickness= 0.500pt
\setplotsymbol ({\thinlinefont .})
\plot 12.541 20.479 13.335 23.019 /
%
% arrow head
%
\plot 13.320 22.757 13.335 23.019 13.199 22.795 /
%
%
% Fig POLYLINE object
%
\linethickness= 0.500pt
\setplotsymbol ({\thinlinefont .})
\putrule from 12.859 20.003 to 15.716 20.003
%
% arrow head
%
\plot 15.462 19.939 15.716 20.003 15.462 20.066 /
%
%
% Fig POLYLINE object
%
\linethickness= 0.500pt
\setplotsymbol ({\thinlinefont .})
\putrule from 13.811 23.495 to 16.616 23.495
%
% arrow head
%
\plot 16.362 23.431 16.616 23.495 16.362 23.558 /
%
%
% Fig POLYLINE object
%
\linethickness= 0.500pt
\setplotsymbol ({\thinlinefont .})
\plot  4.128 18.891  7.990 21.484 /
%
% arrow head
%
\plot  7.815 21.290  7.990 21.484  7.744 21.395 /
%
%
% Fig POLYLINE object
%
\linethickness= 0.500pt
\setplotsymbol ({\thinlinefont .})
\plot  7.779 16.986 13.070 23.072 /
%
% arrow head
%
\plot 12.952 22.838 13.070 23.072 12.856 22.922 /
%
%
% Fig POLYLINE object
%
\linethickness= 0.500pt
\setplotsymbol ({\thinlinefont .})
\putrule from  8.096 16.669 to 10.954 16.669
%
% arrow head
%
\plot 10.700 16.605 10.954 16.669 10.700 16.732 /
%
%
% Fig POLYLINE object
%
\linethickness= 0.500pt
\setplotsymbol ({\thinlinefont .})
\putrule from 11.959 16.669 to 14.764 16.669
%
% arrow head
%
\plot 14.510 16.605 14.764 16.669 14.510 16.732 /
%
%%%%  Cut
\linethickness= 0.500pt
\setplotsymbol ({\thinlinefont .})
\putrule from 11.2 23.0 to 14.0 16.3
%
\sf
%
% Fig TEXT object
%
\put{\yel{1}} [lB] at  3.493 18.415
%
% Fig TEXT object
%
\put{\yel{2}} [lB] at  7.303 16.457
%
% Fig TEXT object
%
\put{\yel{4}} [lB] at 11.271 16.457
%
% Fig TEXT object
%
\put{\yel{7}} [lB] at 15.081 16.457
%
% Fig TEXT object
%
\put{\yel{3}} [lB] at  8.255 21.484
%
% Fig TEXT object
%
\put{\yel{6}} [lB] at 13.176 23.283
%
% Fig TEXT object
%
\put{\yel{9}} [lB] at 17.092 23.283
%
% Fig TEXT object
%
\put{\yel{5}} [lB] at 12.224 19.791
%
% Fig TEXT object
%
\put{\yel{8}} [lB] at 16.034 19.791
\linethickness=0pt
\putrectangle corners at  1.587 24.638 and 17.664 16.150
%
%\pause
%\linethickness=2pt
%\setplotsymbol ({.})
%\color{orange}
\putrule from  11.2 24.0 to 14.0 16.2
%
\endpicture
%}

	\end{center}
\end{small}	
	%
	Soit $\yel{\cB} = \cA\setminus\cL$ et $\yel{\cG} = \{V_j,\, j \in\cB\}$.

\end{frame}

%%%%%%%%%%%%%%%%%%%%%%%%%%%%%%

\begin{frame}
	\frametitle{Estimateur CMC}
	
	\[
	\yel{X_\rme} = P[T > x\mid \{V_j,\, j \in\cB\}].
	\]
	On le calcule comme suit.
	
	Pour chaque $\yel{l}\in\cL$, allant disons de $\yel{a_l}$ \`a $\yel{b_l}$,
	calculer la longueur $\yel{\alpha_l}$ du plus long chemin de la source 
	\`a $a_l$,
	puis la longueur $\yel{\beta_l}$ du plus long chemin de $b_l$ au puits.
	Aucun chemin passant par $l$ n'est plus long que $x$ ssi
	$\alpha_l + V_l + \beta_l \le x$.
	
	
	Conditionnellement \`a $\{V_j,\, j \in\cB\}$, on a cette condition 
	avec probabilit\'e $P[V_l \le x - \alpha_l - \beta_l] 
	= F_l[x - \alpha_l - \beta_l]$.
	
	
	Puisque les $V_l$ sont ind\'ependants, on obtient
	\[
	\yel{X_\rme} = 1 - \prod_{l\in\cL} F_l[x - \alpha_l - \beta_l].
	\]
	Remarque: cet estimateur peut \^etre moins co\^uteux \`a calculer que $X$.

\end{frame}

%%%%%%%%%%%%%%%%%%%%%%%%%%%%%%

\begin{frame}
\frametitle{Estimateur CMC}
	
	Avec l'exemple tir\'e de Avramidis et Wilson (1998), cette m\'ethode
	r\'eduit la variance environ par un facteur 4.
	
	
	Dans cet exemple, on peut aussi utiliser les $V_j$, et peut-\^etre
	les $V_j^2$, comme variables de contr\^ole. 
	On sait que ces variables sont corr\'el\'ees avec $X_\rme$.
	
\end{frame}

%%%%%%%%%%%%%%%%%%%%%%%%%%%%%%

\begin{frame}
	\frametitle{Exemple: banque}
	
	
	Soit $\yel{B}$ le nombre d'abandons dans une journ\'ee.
	On peut estimer $E[B]$ par $B$.
	
	
	Supposons maintenant que l'on rend les \emph{abandons invisibles}.\\
	$\yel{\cG}$ represente tout le reste de l'information.
	
	\mbox{}
	
	L'estimateur CMC est $B_{\rme} = E[B \mid \cG]$.
	
	
	Soit $\yel{\lambda(t)}$ le taux d'arriv\'ee au temps $t$;\\
	$\yel{N(t)}$ le nombre de clients dans le syst\`eme au temps $t$;\\
	$\yel{P(t)} = p_{N(t)}$.
	
	
	Le processus d'arriv\'ee conditionnel (invisible) des clients qui 
	abandonnent est un processus de Poisson de taux 
	$\{\lambda(t) P(t),\, t\ge 0\}$. 
	
\end{frame}

\end{document}
%%%%%%%%%%%%%%%%%%%%%%%%%%%%%%%%%%%%%%%%%%%%%%%%%%%%%%%%%%%%%%%%%%%%%%
