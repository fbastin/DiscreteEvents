\documentclass[t,usepdftitle=false]{beamer}
\usepackage[utf8]{inputenc}
\usetheme{Warsaw}
\usepackage{xcolor}

\usepackage{etex}
\usepackage{pictex}
\usepackage{tikz}
\usetikzlibrary{shapes,arrows}
\usepackage{pgfplots}

\usepackage{amsmath}
\usepackage{amscd}
\usepackage{pstricks, pst-node}

\usepackage{times}

\usepackage{ulem}

\usepackage{amsmath, amsthm}
\usepackage{listings}

% \setbeamercovered{transparent}
%\usecolortheme{crane}
\title[IFT3245]{IFT 3245\\Simulation et modèles}
\author[Fabian Bastin]{Fabian Bastin\\DIRO\\Université de Montréal}
\date{Automne 2016}

\def\be{\boldsymbol{e}}
\def\bh{\boldsymbol{h}}
\def\bu{\boldsymbol{u}}
\def\bv{\boldsymbol{v}}
\def\bx{\boldsymbol{x}}
\def\bA{\boldsymbol{A}}
\def\bP{\boldsymbol{P}}
\def\bX{\boldsymbol{X}}
\def\bY{\boldsymbol{Y}}
\def\bmu{\boldsymbol{\mu}}
\def\bSigma{\boldsymbol{\Sigma}}
\def\bzero{\boldsymbol{0}}

\def\eqas{\overset{\mbox{p.s.}}{=}}

\def\cH{\mathcal{H}}
\def\cJ{\mathcal{J}}
\def\cS{\mathcal{S}}

\def\rit{\mathcal{R}}
\def\NN{\mathcal{N}}
\def\RR{\mathcal{R}}
\def\ZZ{\mathcal{Z}}

\def\eps {$< 10^{-15}$}
\def\epsm {$> 1-10^{-15}$}

\def\MSE{\mbox{MSE}}
\def\RE{\mbox{RE}}
\def\eff{\mbox{Eff}}
\def\Var{\mbox{Var}}
\def\Cov{\mbox{Cov}}
\def\var{\mbox{var}}

\def\iid{i.i.d.}
\def\toas{\mbox{ to as }}
\def\To{\overset{D}{\to}}

\newtheorem{thm}{Theorème}
\newtheorem{defn}{Définition}
\newtheorem{coro}{Corollaire}
\newtheorem{prop}{Proposition}
\newtheorem{exe}{Exemple}

\begin{document}
\frame{\titlepage}

\begin{frame}
\frametitle{Intervalle basé sur une simulation unique}

Supposons ici que le processus $\{{C_j},\, j\ge 1\}$ est stationnaire,
avec $E[C_j]={\mu}$, (e.g., la partie ``échauffement'' a déjà
été enlevée.)

\mbox{}

Si nous estimons $\mu$ par $\bar C_n$, comment pouvons-nous estimer $\Var[\bar C_n]$?

\mbox{}

Deux techniques particulières:
\begin{itemize}
\item
moyennes par lots;
\item
simulation regénérative.
\end{itemize}

\end{frame}

\begin{frame}
\frametitle{Moyennes par lots (``batch means'')}

C'est l'approche la plus simple et la plus populaire pour les
systèmes complexes.
L'idée de base consiste à regrouper les ${n}$ observations en ${k}$
\emph{lots} de taille ${\ell} = n/k$.
Nous pouvons alors écrire
\begin{itemize}
\item
moyenne pour le lot $i$:
$$
  {X_i} = {1\over \ell} \sum_{j=\ell (i-1)+1}^{\ell i} C_j.
$$
\item
moyenne globale:
$$
 {\bar X_k} = {1\over k} \sum_{i=1}^k X_i = \bar C_n.
$$
\end{itemize}

\end{frame}

\begin{frame}
\frametitle{Moyennes par lots (``batch means'')}

Si $\ell$ est grand, on s'attend à ce que les $X_i$ soient très 
peu corrélés et suivent à peu près la loi normale.

\mbox{}


Sous certaines conditions additionnelles, il est possible de prouver
que c'est le cas pour $\ell \rightarrow \infty$, et $k$ fixé.
Nous pouvons alors calculer un intervalle de confiance en faisant
l'hypothèse que les $X_i$ sont \iid{} $N(\mu, \sigma_x^2)$,
et que donc $\sqrt{k}(\bar X_k - \mu)/ S_k \sim$ Student$(k-1)$.

\mbox{}

Il est souvent recommandé de choisir $\ell$ le plus grand possible
et $k\le 30$.
Un $k$ plus petit diminue le \emph{bias} $E[S_k^2/k] - \Var[\bar X_k]$,
mais augmente la variance de $S_k^2/k$.

\end{frame}

\begin{frame}
\frametitle{Simulation regénérative}

L'idée consiste à écrire la moyenne sur horizon infini comme un rapport
de deux espérances sur horizon fini.

\mbox{}

\begin{defn}{Processus regénératif}
\index{processus!regeneratif@regénératif}
Un processus stochastique $\{{Y(t)},\; t\ge 0\}$ 
est \emph{regénératif} (au sens classique) s'il existe
une variable aléatoire ${\tau_1} > 0$ telle que 
$\{Y(\tau_1+t),\; t\ge 0\}$ est stochastiquement équivalent à
$\{Y(t),\; t\ge 0\}$ et indépendant de $\tau_1$ et de 
$\{Y(t),\; t<\tau_1\}$.
\end{defn}

La variable aléatoire $\tau_1$ est un \emph{instant de regénération}.
La trajectoire du processus sur l'intervalle de temps $(0,\tau_1]$
est un \emph{cycle regénératif}, et si $E[\tau_1] < \infty$, le processus est dit \emph{récurrent positif}.

\mbox{}

L'adaptation au cas discret est évidente.

\end{frame}

\begin{frame}
\frametitle{Simulation regénérative}

Si $\{Y(t),\; t\ge 0\}$ est regénératif, alors 
$\{Y(\tau_1+t),\; t\ge 0\}$ est aussi regénératif avec un
instant de regénération ${\tau_2}$, etc.

\mbox{}

On a ainsi une suite infinie d'instants de regénération
$0 =\tau_0 < \tau_1 < \tau_2 \dots$ et de cycles regénératifs \iid

\mbox{}

Parfois, le premier cycle est différent des autres:
le processus est regénératif \emph{avec délai}
(à partir de $\tau_1$), mais l'effet est négligeable à long terme.

\mbox{}

Si $\{Y(t),\; t\ge 0\}$ est regénératif aux instants 
$\tau_1,\tau_2,\dots$ et si ${C(t)} = f(Y(t))$, alors
$\{C(t),\, t\ge 0\}$ est aussi regénératif aux mêmes instants.

\end{frame}

\begin{frame}
\frametitle{Simulation regénérative: exemple}

File $GI/G/1$\index{GI/G/1@$GI/G/1$} stable.

\mbox{}

Si $W_1=0$, si les v.a.\ $S_i-A_i$ sont \iid{} et  $E[S_i - A_i] < 0$, alors
$\{{W_i},\, i\ge 1\}$ est regénératif et on peut prendre pour instants 
de regénération les époques $i$ où ${W_i=0}$.

\mbox{}

Le processus $\{{Q(t)},\; t\ge 0\}$ regénère aussi, aux instants
où un client \emph{arrive} dans un système vide.

\mbox{}

Mais peut-on prendre les instants où le système \emph{se vide} comme instants
de regénération? La réponse est positive seulement dans le cas $M/G/1$.

\end{frame}

\begin{frame}
\frametitle{Simulation regénérative: exemple}

Centre d'appels

\mbox{}

Supposons qu'un centre d'appels opère pour une suite infinie de jours \iid.
Il ouvre à 8h et ferme à 21h.

\mbox{}

Soit ${t}$ le temps écoulé depuis le premier jour à minuit,
en heures, et ${Q(t)} =$ le nombre d'appels dans la file au temps $t$.
%  Si ${t_0} \in [0,8]\cup[21,24]$,
Le processus $\{Q(t),\; t\ge 0\}$ regénère à 
${\tau_j} = 24j$ pour $j=1,2,\dots$.

\mbox{}

Si ${X_j}=$ nombre d'appels reçus au jour $j$,
$\{X_j,\, j\ge 0\}$ regénère à ${\tau_j} = j$ pour tout $j$ 
(\emph{processus de renouvellement}).

\mbox{}

Il en est de même pour $\{Z_j,\, j\ge 0\}$ si 
${Z_j}=$ nombre d'abandons au jour $j$.

\end{frame}

\begin{frame}
\frametitle{Théorème du renouvellement avec gains}

Soit $\{{C_i},\;i\ge 0\}$ un processus regénératif aux instants 
${\tau_j}$, $j\ge 0$, et $N(t)$ le nombre d'événements qui se sont produits au cours de l'intervalle $[0,t]$.

\mbox{}

Posons 
\[
  {X_j} = V_{N(\tau_j)}-V_{N(\tau_{j-1})},
\]
le coût pour le cycle $j$, et
\[
  {Y_j} =  \tau_{j} - \tau_{j-1},
\]
la durée du cycle $j$.

\end{frame}

\begin{frame}
\frametitle{Théorème du renouvellement avec gains}

\begin{thm}
Si $E[Y_j] > 0$ et $E[|X_j|] < \infty$, alors,
\begin{eqnarray*}
  \bar v &\overset{\mbox{def}}{=}& \lim_{t\to\infty} {E[V_{N(t)}]\over t}
         ~=~ {E[X_j]\over E[Y_j]}
                \qquad\mbox{(version \emph{espérance}),}
\end{eqnarray*}
et
\begin{eqnarray*}
  \bar v &\eqas& \lim_{t\to\infty} {V_{N(t)}\over t}
                \qquad\mbox{(version \emph{trajectoire})}.
\end{eqnarray*}
\end{thm}

\end{frame}

\begin{frame}
\frametitle{Théorème du renouvellement avec gains}

Une fois le nombre de cycles ${n}$ fixé, nous retrouvons le problème d'estimation du quotient
$\bar v = E[X_j]/ E[Y_j]$ à partir des observations \iid{} $(X_1,Y_1),\dots,(X_n,Y_n)$.,

\mbox{}

Toutefois, pour une durée de la simulation à ${t}$, le nombre de cycles
${M(t)} = \sup\{n \ge 0: \tau_n \le t\}$ est aléatoire.

\mbox{}

Lorsque nous atteignons $t$, nous pouvons terminer le cycle en cours (nous aura $M(t)+1$ cycles), ou
ignorer le cycle en cours (nous en aurons $M(t)$).
La variance est dans $O(1/t)$ dans les deux cas.
Le biais sur $\bar v$ est dans $O(1/t^2)$ pour la première approche, et $O(1/t)$ pour la seconde.

\end{frame}

\begin{frame}
\frametitle{Théorème du renouvellement avec gains}

\begin{thm}
Sous les conditions du théorème de la limite centrale pour un quotient, quand $t\to\infty$, % (Th. 3.6),
\begin{align*}
 {\sqrt{M(t)} (\hat v_{M(t)} - v) \over \hat\sigma_{g,M(t)}} & \To
 {\sqrt{t/\bar Y_{M(t)}} (\hat v_{M(t)} - v) \over \hat\sigma_{g,M(t)}} \\
 &\To {\sqrt{t/E[Y_1]} (\hat v_{M(t)} - v) \over \hat\sigma_{g,M(t)}} \\
  &\To N(0,1),
%\label{eq:regeneration}
\end{align*}
où $\sigma_g$ est défini comme dans le cadre du théorème delta.

\mbox{}

Le résultat demeure valide si nous remplaçons $M(t)$ par $M(t)+1$.
\end{thm}

\end{frame}

\end{document}