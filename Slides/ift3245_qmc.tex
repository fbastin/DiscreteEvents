\documentclass[t,usepdftitle=false]{beamer}
\usepackage[utf8]{inputenc}
\usetheme{Warsaw}
\usepackage{xcolor}

\usepackage{etex}
\usepackage{pictex}
\usepackage{tikz}
\usetikzlibrary{shapes,arrows}
\usepackage{pgfplots}

\usepackage{graphicx}

\usepackage{amsmath}
\usepackage{amscd}
\usepackage{pstricks, pst-node}

\usepackage{times}

\usepackage{ulem}

\usepackage{amsmath, amsthm}
\usepackage{listings}

% \setbeamercovered{transparent}
%\usecolortheme{crane}
\title[IFT3245]{IFT 3245\\Simulation et modèles\\Introduction au quasi-Monte Carlo}
\author[Fabian Bastin]{Fabian Bastin\\DIRO\\Université de Montréal}
\date{Automne 2016}

\def\be{\boldsymbol{e}}
\def\bh{\boldsymbol{h}}
\def\bu{\boldsymbol{u}}
\def\bv{\boldsymbol{v}}
\def\bx{\boldsymbol{x}}
\def\bA{\boldsymbol{A}}
\def\bC{\boldsymbol{C}}
\def\bP{\boldsymbol{P}}
\def\bU{\boldsymbol{U}}
\def\bX{\boldsymbol{X}}
\def\bY{\boldsymbol{Y}}

\def\bone{\boldsymbol{1}}

\def\q{\quad}

\def\bbeta{\boldsymbol{\beta}}
\def\bmu{\boldsymbol{\mu}}
\def\bnu{\boldsymbol{\nu}}
\def\bSigma{\boldsymbol{\Sigma}}
\def\bzero{\boldsymbol{0}}

\def\eqas{\overset{\mbox{p.s.}}{=}}

\def\cA{\mathcal{A}}
\def\cH{\mathcal{H}}
\def\cI{\mathcal{I}}
\def\cJ{\mathcal{J}}
\def\cL{\mathcal{L}}
\def\cN{\mathcal{N}}
\def\cS{\mathcal{S}}

\def\rit{\mathcal{R}}
\def\NN{\mathcal{N}}
\def\RR{\mathcal{R}}
\def\ZZ{\mathcal{Z}}

\def\eps {$< 10^{-15}$}
\def\epsm {$> 1-10^{-15}$}

\def\MSE{\mbox{MSE}}
\def\RE{\mbox{RE}}
\def\eff{\mbox{Eff}}
\def\Var{\mbox{Var}}
\def\Cov{\mbox{Cov}}
\def\var{\mbox{var}}

\def\iid{i.i.d.}
\def\toas{\mbox{ to as }}
\def\To{\overset{D}{\to}}

\newtheorem{assumption}{Hypothèse}
\newtheorem{thm}{Theorème}
\newtheorem{defn}{Définition}
\newtheorem{coro}{Corollaire}
\newtheorem{prop}{Proposition}
\newtheorem{exe}{Exemple}

\begin{document}
\frame{\titlepage}

\begin{frame}
\frametitle{M\'ethodes quasi-Monte Carlo (QMC)}

(Tiré de Pierre L'Ecuyer)

On veut estimer 
$$
{\mu} = \int_{[0,1)^t} f(\bu) d\bu
$$
par la moyenne de $f$ sur l'ensemble
${P_n} =\{\bu_1,\dots,\bu_{n}\} \subset [0,1)^{t}$:
$$
{\bar\mu_n} = {1\over n} \sum_{i=1}^{n} f(\bu_i).
$$

\emph{Id\'ee de QMC}: 
Choisir $P_n = \{\bu_1,\dots,\bu_{n}\}$ plus uniform\'ement distribu\'e
qu'un ensemble de points au hasard.\\
Ensembles (et suites) de points \`a \emph{faible discrépance}, ou
\emph{hautement uniformes}.

\mbox{}

\emph{MC}: erreur ${E_n} = \bar\mu_n - \mu$ al\'eatoire.\\
\emph{QMC classique}: $E_n$ est d\'eterministe.
 
\end{frame}

\begin{frame}
\frametitle{Questions}

Est-ce que $E_n\to 0$ plus vite avec QMC que MC quand $n\to\infty$?

\mbox{}

Comment construire $P_n$ et mesurer son uniformit\'e? \\
Facile en 1 dimension.  Mais en plusieurs dimensions?\\
Pour $t$ grand, il faut trop de points pour vraiment bien remplir l'espace.\\
Est-ce que QMC peut quand m\^eme fonctionner?  Pourquoi?

\end{frame}

\begin{frame}
\frametitle{Dimension effective}

Dans plusieurs cas, $f$ peut s'approximer assez bien par une somme de
fonctions $f_I$ dont chacune ne d\'epend que de peu de coordonn\'ees de $\bu$:
\begin {eqnarray*}
f(\bu) &=& f(u_1,\dots,u_t)\\
&=& \sum_{{I} \subseteq \{1,\dots,t\}} {f_I}(\bu) 
= \mu + \sum_{i=1}^t \dot f_{\{i\}}(u_i) 
+ \sum_{i,j=1}^t \dot f_{\{i,j\}}(u_i, u_j) + \cdots 
\end {eqnarray*}
o\`u $f_I$ ne d\'epend que de $\{u_i,\, i\in I\}$.

\mbox{}

Il suffit alors que les \emph{projections} correspondantes de $P_n$ 
soient bien uniformes.

\end{frame}

\begin{frame}
\frametitle{Cas simple: une dimension ($t=1$)}

Solutions \'evidentes:
\[
P_n = \ZZ_n/n = \{0,\, 1/n,\, \dots, (n-1)/n\}\]
ou encore
\[
P'_n = \{1/(2n),\, 3/(2n),\, \dots, (2n-1)/(2n)\}.
\]
Si on permet des \emph{poids} diff\'erents pour les $f(\bu_i)$, on a aussi
la \emph{r\`egle du trap\`eze}, 
\[
{1\over n} \left[{f(0) + f(1)\over 2} + \sum_{i=1}^{n-1} f(i/n) \right],
\]
pour laquelle $|E_n| = O(n^{-2})$ si $f''$ est born\'ee, 
ou la \emph{r\`egle de Simpson},
\[
\frac{f(0) + 4f(1/n) + 2f(2/n) + \cdots 
	+ 2f((n-2)/n) + 4f((n-1)/n) + f(1)}{3n},
\]
qui donne $|E_n| = O(n^{-4})$ si $f^{(4)}$ est born\'ee, etc.

\end{frame}

\begin{frame}
\frametitle{Cas simple: une dimension ($t=1$)}

Ici, on se restreint \`a des \emph{poids \'egaux}.
Plus tard, on va randomizer $P_n$ pour mieux estimer l'erreur et dans ce
contexte, les poids \'egaux sont optimaux.

\end{frame}

\begin{frame}
\frametitle{$t > 1$ dimensions}

Solution simpliste en: \emph{grille rectangulaire}
\[
P_n = \{(i_1/d,\dots,i_t/d) \mbox{ tels que } 0\le i_j < d \ \ \forall j\},
\]
o\`u $n = d^t$. \q

\mbox{}

Mais devient vite inutilisable quand $t$ augmente.

\mbox{}

Et on \emph{perd des points} dans les projections: chaque projection le long d'un axe correspond à un ensemble de points.

\end{frame}

\begin{frame}
\frametitle{$t > 1$ dimensions: illustration}

\begin {center}
\hskip 0.000001 cm 
\beginpicture
\setcoordinatesystem units <7cm,7cm>
\setplotarea x from 0 to 1, y from 0 to 1
\axis bottom
ticks withvalues     0.0  $u_n$  1.0 / at 0.0 0.5 1.0 / / 
\axis left
ticks withvalues     0.0  $u_{n+1}$~~  1.0 / at 0.0 0.5 1.0 / / 
\axis top /  \axis right /
%
\multiput {\large\bf .} at
0.000  0.000   
0.000  0.125      
0.000  0.250      
0.000  0.375      
0.000  0.500      
0.000  0.625      
0.000  0.750      
0.000  0.875      
%
0.125  0.000
0.125  0.125      
0.125  0.250
0.125  0.375
0.125  0.500
0.125  0.625
0.125  0.750
0.125  0.875
%
0.250  0.000
0.250  0.125      
0.250  0.250
0.250  0.375
0.250  0.500
0.250  0.625
0.250  0.750
0.250  0.875
%
0.375  0.000
0.375  0.125      
0.375  0.250
0.375  0.375
0.375  0.500
0.375  0.625
0.375  0.750
0.375  0.875
%
0.500  0.000
0.500  0.125      
0.500  0.250
0.500  0.375
0.500  0.500
0.500  0.625
0.500  0.750
0.500  0.875
%
0.625  0.000
0.625  0.125      
0.625  0.250
0.625  0.375
0.625  0.500
0.625  0.625
0.625  0.750
0.625  0.875
%
0.750  0.000
0.750  0.125      
0.750  0.250
0.750  0.375
0.750  0.500
0.750  0.625
0.750  0.750
0.750  0.875
%
0.875  0.000
0.875  0.125      
0.875  0.250
0.875  0.375
0.875  0.500
0.875  0.625
0.875  0.750
0.875  0.875
/ \endpicture
\end {center}

\end{frame}

\begin{frame}
\frametitle{$t > 1$ dimensions}

\emph{Id\'ee}: on voudrait construire $P_n$ de mani\`ere \`a ce que 
chaque projection unidimensionelle soit $\{0,\, 1/n,\, \dots, (n-1)/n\}$.

\mbox{}

Pour cela, on doit \'enum\'erer ces $n$ valeurs dans un ordre diff\'erent
pour chaque coordonn\'ee. (Autrement tous les points seront sur une diagonale.)

\end{frame}

\begin{frame}
\frametitle{$t > 1$ dimensions}

\begin {center}
\hskip 0.000001 cm 
\beginpicture
\setcoordinatesystem units <6cm,6cm>
\setplotarea x from 0 to 1, y from 0 to 1
\axis bottom
ticks withvalues     0.0  $u_n$  1.0 / at 0.0 0.5 1.0 / / 
\axis left
ticks withvalues     0.0  $u_{n+1}$~~  1.0 / at 0.0 0.5 1.0 / / 
\axis top /  \axis right /
%
\multiput {\large\bf .} at
0.000  0.000   
0.125  0.125      
0.250  0.250
0.375  0.375
0.500  0.500
0.625  0.625
0.750  0.750
0.875  0.875
/ \endpicture
\end {center}

i.e., on cherche une \emph{permutation} de $\ZZ_n$ pour chacune des 
$t$ coordonn\'ees pour que $P_n$ soit tr\`es uniforme
sur $[0,1)^t$.

\end{frame}

\begin{frame}
\frametitle{$t > 1$ dimensions}

\emph{Exemple}: Soient $n = 2^{8} = 256$ et $t=2$.
Prenons les points (en binaire):
%
\begin{center}
	\begin{tabular}{rll}
		\hline
		$i$ & ${u_{1,i}}$ & ${u_{2,i}}$ \\
		\hline
		0 &  .00000000  & .0 \\
		1 &  .00000001  & .1 \\
		2 &  .00000010  & .01 \\
		3 &  .00000011  & .11 \\
		4 &  .00000100  & .001 \\
		5 &  .00000101  & .101 \\
		6 &  .00000110  & .011 \\
		\vdots & \vdots & \vdots \\
		254 &  .11111110  & .01111111 \\
		255 &  .11111111  & .11111111 \\
		\hline
	\end{tabular}
\end{center}

\end{frame}

\begin{frame}[fragile]
\frametitle{Exemple: $n = 2^{8} = 256$ and $t=2$}

\begin{footnotesize}
\vbox{
	\begin {center}
	\hskip 0.000001 cm 
	\beginpicture
\setcoordinatesystem units <7cm,7cm> 
%\unitlength=5cm 
%	\setcoordinatesystem units <14cm,14cm>
	\setplotarea x from 0 to 1, y from 0 to 1
	\axis bottom
	ticks withvalues     0.0  $u_n$  1.0 / at 0.0 0.5 1.0 / / 
	\axis left
	ticks withvalues     0.0  $u_{n+1}$~~  1.0 / at 0.0 0.5 1.0 / / 
	\axis top /  \axis right /
	%
	%\setdots <1.0pt>
	\input vdc2-256.tex
%	\hskip -0.24cm
	\linethickness = .03pt
	\axis bottom invisible ticks width <.10pt> length <0pt> 
	andacross quantity 17 /
	\axis left   invisible ticks width <.10pt> length <0pt> 
	andacross quantity 17 /
	%
	\axis bottom invisible ticks width <.10pt> length <0pt> 
	andacross quantity 9 /
	\axis left   invisible ticks width <.10pt> length <0pt> 
	andacross quantity 33 /
	%
	\axis bottom invisible ticks width <.10pt> length <0pt> 
	andacross quantity 65 /
	\axis left   invisible ticks width <.10pt> length <0pt> 
	andacross quantity 5 /
	\endpicture
	\end {center}
}
\end{footnotesize}

\end{frame}

\begin{frame}[fragile]
\frametitle{Exemple: $n = 2^{8} = 256$ and $t=2$}

Ces points forment un $(0,8,2)$-r\'eseau en base 2 (explication de
cette appellation ultérieurement).

\mbox{}

En g\'en\'eral, on peut prendre ${n = 2^k}$ points.

\mbox{}

Si on partitionne $[0,1)^2$ en rectangles de tailles $2^{-k_1}$ par $2^{-k_2}$
o\`u $k_1 + k_2 \le k$, chaque rectangle contiendra exactement le m\^eme
nombre de points.

\mbox{}

Ce type de $P_n$ est un cas particulier d'un \emph{r\'eseau digital} en base 2.

\end{frame}

\begin{frame}[fragile]
\frametitle{``Lattice'' (r\'eseau) avec $n = 101$}

\vbox{
	\begin{center}
		\hskip 0.0000001cm
		\beginpicture
		\setcoordinatesystem units <5cm,5cm>
		\setplotarea x from 0 to 1, y from 0 to 1
		\axis bottom
		ticks withvalues     0.0  $u_n$  1.0 / at 0.0 0.5 1.0 / / 
		\axis left
		ticks withvalues     0.0  $u_{n+1}$~~  1.0 / at 0.0 0.5 1.0 / / 
		\axis top /  \axis right /
		\multiput {\large\bf .} at
		0.0      0.0
		0.11881  0.42574
		0.42574  0.10891
		0.10891  0.30693
		0.30693  0.68317
		0.68317  0.19802
		0.19802  0.37624
		0.37624  0.51485
		0.51485  0.17822
		0.17822  0.13861
		0.13861  0.66337
		0.66337  0.96040
		0.96040  0.52475
		0.52475  0.29703
		0.29703  0.56436
		0.56436  0.77228
		0.77228  0.26733
		0.26733  0.20792
		0.20792  0.49505
		0.49505  0.94059
		0.94059  0.28713
		0.28713  0.44554
		0.44554  0.34653
		0.34653  0.15842
		0.15842  0.90099
		0.90099  0.81188
		0.81188  0.74257
		0.74257  0.91089
		0.91089  0.93069
		0.93069  0.16832
		0.16832  0.01980
		0.01980  0.23762
		0.23762  0.85149
		0.85149  0.21782
		0.21782  0.61386
		0.61386  0.36634
		0.36634  0.39604
		0.39604  0.75248
		0.75248  0.02970
		0.02970  0.35644
		0.35644  0.27723
		0.27723  0.32673
		0.32673  0.92079
		0.92079  0.04950
		0.04950  0.59406
		0.59406  0.12871
		0.12871  0.54455
		0.54455  0.53465
		0.53465  0.41584
		0.41584  0.99010
		0.99010  0.88119
		0.88119  0.57426
		0.57426  0.89109
		0.89109  0.69307
		0.69307  0.31683
		0.31683  0.80198
		0.80198  0.62376
		0.62376  0.48515
		0.48515  0.82178
		0.82178  0.86139
		0.86139  0.33663
		0.33663  0.03960
		0.03960  0.47525
		0.47525  0.70297
		0.70297  0.43564
		0.43564  0.22772
		0.22772  0.73267
		0.73267  0.79208
		0.79208  0.50495
		0.50495  0.05941
		0.05941  0.71287
		0.71287  0.55446
		0.55446  0.65347
		0.65347  0.84158
		0.84158  0.09901
		0.09901  0.18812
		0.18812  0.25743
		0.25743  0.08911
		0.08911  0.06931
		0.06931  0.83168
		0.83168  0.98020
		0.98020  0.76238
		0.76238  0.14851
		0.14851  0.78218
		0.78218  0.38614
		0.38614  0.63366
		0.63366  0.60396
		0.60396  0.24752
		0.24752  0.97030
		0.97030  0.64356
		0.64356  0.72277
		0.72277  0.67327
		0.67327  0.07921
		0.07921  0.95050
		0.95050  0.40594
		0.40594  0.87129
		0.87129  0.45545
		0.45545  0.46535
		0.46535  0.58416
		0.58416  0.00990
		0.00990  0.11881
		0.11881  0.42574
		0.42574  0.10891
		/
		\arrow <5pt> [0.5, 1.0] from 0.0 0.0 to 0.009901 0.118812
		\arrow <5pt> [0.5, 1.0] from 0.0 0.0 to 0.0 1.0 
		\endpicture
\end{center}
		
\emph{Exemple jouet}: LCG avec $m=101$ et $a=12$.\\
Les $\bu_i$ sont tous les multiples entiers de $(1/101,\, 12/101)$, modulo 1.
}

\end{frame}

\begin{frame}[fragile]
\frametitle{LCG avec $t=2$ et $n=1021$}

\begin{center}
	\hskip 0.0000001cm
	\beginpicture
	\setcoordinatesystem units <7cm,7cm>
	\setplotarea x from 0 to 1, y from 0 to 1
	\axis bottom
	ticks withvalues     0.0  $u_n$  1.0 / at 0.0 0.5 1.0 / / 
	\axis left
	ticks withvalues     0.0  $u_{n+1}$~~  1.0 / at 0.0 0.5 1.0 / / 
	\axis top /  \axis right /
	\input lcg1021-90.tex
	\endpicture
\end{center}

\end{frame}

\begin{frame}
\frametitle{Réseaux digitaux}

Les r\'eseaux digitaux (``digital nets'') et les r\`egles de r\'eseaux
(``lattice rules'') se g\'en\'eralisent \`a un $t$ arbitraire.

\mbox{}

Par exemple, on peut d\'efinir:
\[
P_n = \{\bu_i = (i \bv \mod n)/n,\; i=0,\dots,n-1\}
\]
pour $\bv = (1,v_1,\dots,v_{t-1})$ bien choisi.

\mbox{}

Pour un r\'eseau digital en base 2, il suffit de choisir $t$ permutations
de $\{0,1,\dots,2^k-1\}$.

\mbox{}

On peut aussi avoir $t=\infty$ et/ou $n = \infty$ (\emph{suite infinie} de points).

\end{frame}

\begin{frame}
	\frametitle{Discr\'epance}
	
	Comment mesurer l'uniformit\'e d'un ensemble de points?\\
	\emph{Id\'ee:} pour ${B}$ un sous-ensemble quelconque de $[0,1)^t$,
	la fraction de $P_n$ qui se trouve dans $B$ devrait \^etre \`a peu 
	pr\`es \'egale au volume de $B$.\\
	Possible pour tout $B$? Non.
	
	
	On peut se limiter \`a une famille ${\cI}$ de bo\^{i}tes $B$, par exemple
	la famille ${\cI_t^*}$ de toutes les bo\^{i}tes rectangulaires align\'ees
	avec les axes et ayant un coin \`a l'origine (les intervalles
	de forme $[\bzero,\bu)$ en $t$ dimensions).\\
	On d\'efinit ${\Delta(B)} = |$(fraction de $P_n$ qui est dans $B$)
	$-$ volume$(B)|$\\
	Mesure de \emph{discr\'epance} pour $P_n$:
	\[
	{D_n^{*}(P_n)} = \sup_{B=[\bzero,\bu)\in \cI_t^*}  \Delta(B).
	\]
	Il y a d'autres possibilit\'es: moyenne $\cL_p$ au lieu du sup,
	autre choix de $\cI$, etc.
	
\end{frame}

\begin{frame}
	\frametitle{In\'egalit\'e de \emph{Koksma-Hlawka}}
	
	Erreur pire cas:
	\[
	|\bar\mu_n - \mu| \le \Vert f-\mu\Vert D_n^*(P_n)
	\]
	o\`u $\Vert f-\mu\Vert$ mesure la \emph{variation de $f$}
	au sens de Hardy et Krause.
	
	
	On connaît des suites infinies de points $P_\infty = \{\bu_1,\bu_2,\dots\}$
	telle que pour $P_n = \{\bu_1,\dots,\bu_n\}$, 
	\ $D_n^*(P_n) = O(n^{-1}(\ln n)^t)$.
	
	\mbox{}
	
	Asymptotiquement, c'est mieux que $O(n^{-1/2})$.\\
	
	Mais: la borne est difficile \`a calculer et ne peut \^etre pratique que
	lorsque $t$ ne d\'epasse pas 8 ou 10.
	
\end{frame}

\begin{frame}
	\frametitle{Estimation de l'erreur}
	
	L'id\'ee est de \emph{randomiser} $P_n$ de mani\`ere \`a:\\
	(1) pr\'eserver la structure et l'uniformit\'e de $P_n$ comme ensemble;\\
	(2) chaque point de $P_n$ randomis\'e suit la loi uniforme sur $[0,1)^t$.
	
\mbox{}
	
	\emph{Exemple.} Une fa\c{c}on de faire cela est le 
	\emph{d\'ecalage al\'eatoire}: g\'en\'erer un seul point 
	$\bU$ uniform\'ement dans $[0,1)^t$ et l'ajouter, modulo 1, \`a 
	\emph{chacun} des points de $P_n$.
	
\mbox{}
	
	On r\'ep\`ete la randomisation ${m}$ fois, ind\'ependamment, et on calcule
	la moyenne ${\bar X_m}$ et la variance ${S_{m}^2}$ des $m$ valeurs de 
	$\bar\mu_n$.\\
	On peut montrer que $E[\bar X_m] = \mu$ et 
	$E[S_{m}^2] = \Var[\bar\mu_n] = m\Var[\bar X_m]$.
	
\mbox{}
	
	Permet de calculer un intervalle de confiance pour $\mu$.\\
	Choix de $m$?  Souvent autour de 10.	
	
\end{frame}

\end{document}
