\documentclass[t,usepdftitle=false]{beamer}
\usepackage[utf8]{inputenc}
\usetheme{Warsaw}
\usepackage{xcolor}

\usepackage{etex}
\usepackage{pictex}
\usepackage{tikz}
\usetikzlibrary{shapes,arrows}
\usepackage{pgfplots}

\usepackage{amsmath}
\usepackage{amscd}
\usepackage{pstricks, pst-node}

\usepackage{times}

\usepackage{ulem}

\usepackage{amsmath, amsthm}
\usepackage{listings}

\usepackage[french]{babel}

% \setbeamercovered{transparent}
%\usecolortheme{crane}
\title[IFT3245]{IFT 3245\\Simulation et modèles}
\author[Fabian Bastin]{Fabian Bastin\\DIRO\\Université de Montréal}
\date{Automne 2016}

\def\be{\boldsymbol{e}}
\def\bh{\boldsymbol{h}}
\def\bu{\boldsymbol{u}}
\def\bv{\boldsymbol{v}}
\def\bx{\boldsymbol{x}}
\def\bA{\boldsymbol{A}}

\def\cH{\mathcal{H}}
\def\cJ{\mathcal{J}}
\def\cS{\mathcal{S}}

\def\rit{\mathcal{R}}
\def\RR{\mathcal{R}}
\def\ZZ{\mathcal{Z}}

\def\eps {$< 10^{-15}$}
\def\epsm {$> 1-10^{-15}$}

\begin{document}
\frame{\titlepage}

\begin{frame}
\frametitle{Acceptation-rejet}

Il s'agit de la technique la plus importante après l'inversion, et est
parfois (mais pas toujours!) compatible avec l'utilisation de
techniques de réduction de variance.
Contrairement à la technique d'inversion, elle ne requiert pas de
pouvoir facilement calculer la fonction de répartition, et encore
moins son inverse.

\mbox{}

Nous allons considérer le cas où $X$ est continu (le cas discret est analogue).

\end{frame}

\begin{frame}
\frametitle{Fonction majorante}

Soit $f(x)$ la densité de $X$, et soit $t$ une fonction majorant $f$, i.e. $f (x) \leq t(x)$ pour tout $x$.
On peut normaliser $t$ comme
\[
r(x) = t(x)/a,\mbox{ où }a = \int_{-\infty}^{\infty} t(s)ds.
\]
$r$ est une densité comme son intégration donne 1:
\[
\int_{-\infty}^{\infty} r(x) = \int_{-\infty}^{\infty} t(s)/a ds
= \frac{1}{a} \int_{-\infty}^{\infty} t(s) ds = 1.
\]


\mbox{}

On choisit $t$ de manière à ce que
\begin{enumerate}[(i)]
\item
ce soit facile de générer des v.a. de densité $r$,
\item$a$ soit petit (proche de 1), ou en d'autres termes $t(x)$ est
  proche de $f(x)$.
\end{enumerate}
Le choix de $t$ peut être automatisé.

\end{frame}

\begin{frame}
\frametitle{Méthode d'acceptation-rejet}

Répéter
\begin{itemize}
\item
Générer $Y$ de densité $r(x)$;\\
\item
Générer $U : U (0, 1)$ indépendante de $Y$;\\
\end{itemize}
jusqu'à obtenir $U \leq f (Y )/t(Y )$; retourner Y.

\mbox{}

La variable aléatoire $Y$ retournée est de densité $f$.

\mbox{}

\textit{Convaincu}?
\end{frame}

\begin{frame}
\frametitle{Méthode d'acceptation-rejet}

\begin{proof}
Nous avons tout d'abord, comme $P[U \leq x] = x$,
\begin{align*}
P [Y\mbox{ est accepté }] & = P [ U \leq f (Y)/t(Y) ] \\
%& = \int_{-\infty}^{\infty} P [U \leq f (Y)/t(Y) | Y = y ] r(y)dy \\
& = \int_{-\infty}^{\infty} P [U \leq f (y)/t(y)] r(y)dy \\
& = \int_{-\infty}^{\infty} f (y)/t(y) r(y)dy \\
& = \frac{1}{a}.
\end{align*}
\end{proof}

\end{frame}

\begin{frame}
\frametitle{Méthode d'acceptation-rejet}

\begin{proof}
Similairement,
\begin{align*}
& P [Y \leq x \mbox{ et } Y \mbox{ est acceptée}] \\
& = \int_{-\infty}^{\infty} P [Y \leq x \mbox{ et } U \leq f (Y )/t(Y ) | Y = y ]r(y )dy \\
& = \int_{-\infty}^{\infty} P [y \leq x \mbox{ et } U \leq f (y)/t(y)]
r(y) dy \\
& = \int_{-\infty}^{x} P [U \leq f (y)/t(y)] r(y) dy \\
& = \int_{-\infty}^{x} [f(y)/t(y)]r(y)dy \\
& = F(x)/a 
\end{align*}
\end{proof}

\end{frame}

\begin{frame}
\frametitle{Méthode d'acceptation-rejet}

\begin{proof}
Nous pouvons dès lors écrire
\begin{align*}
P[Y \leq x | Y \mbox{ est accepté}] & =
\frac{P [Y \leq x \mbox{ et } Y \mbox{ est accepté }]}{P [Y\mbox{ est
    accepté }]} \\
& = \frac{F (x)/a}{1/a} = F (x).
\end{align*}
\end{proof}

\end{frame}

\begin{frame}
\frametitle{Méthode d'acceptation-rejet}

\`A chaque tour de boucle, la probabilité d'accepter $Y$ est $1/a$.

\mbox{}

Le nombre de tours de boucle avant l'acceptation est une
v.a. géométrique de paramètre $p = 1/a$.

\mbox{}

Le nombre moyen de tours de boucle par v.a. est donc $a$.

\end{frame}

\begin{frame}
\frametitle{Méthode d'acceptation-rejet: exemple}

Supposons que nous souhaitons tirer des valeurs d'une loi beta de paramètres (4,3), de fonction de densité
\[
f(x) = 60x^3(1 - x)^2, \mbox{ pour } 0 < x < 1.
\]
$F (x)$ est difficile à inverser, mais il est facile d'établir que le maximum de $f$ est $f (0.6)$ = $2.0736$.

\mbox{}

On peut donc prendre $t(x) = 2.0736$ pour $0 < x < 1$.
Dans ce cas $r(x)$ est la densité $U (0, 1)$.

\mbox{}

On peut diminuer $a$ en choisissant une densité $t$ un peu moins
simpliste.

\end{frame}

\begin{frame}
\frametitle{Cas particuliers}

Bien que les méthodes décrites dans les sections précédentes décrivent les techniques majeures pour générer des variables aléatoires multivariées, il existe de nombreuses approches spécifiques aux distributions considérées.

\mbox{}

Ces techniques sont cependant rarement compatibles avec les techniques d'amélioration d'efficacité, et dès lors devraient être employées parcimonieusement.

\mbox{}

Référence utile: Luc Devroye,
\textsl{Non-Uniform Random Variate Generation}, 
\url{http://www.nrbook.com/devroye/}.

\end{frame}

\begin{frame}
\frametitle{Méthode de Box-Muller pour la loi normale}

Génération de variables aléatoires normales en suivant une transformée de coordonnées cartésiennes en coordonnées polaires.

\mbox{}

La méthode proposée en 1958 par Box et Muller; exacte, mais plus lente que l'inversion.

\mbox{}

Son intérêt réside donc d'avantage au niveau théorique qu'au niveau pratique, mais ce type de transformation peut se révéler utile dans d'autres cadres.

\end{frame}

\begin{frame}
\frametitle{Méthode de Box-Muller pour la loi normale}

Considérons un couple $(X,Y)$ de variables aléatoires $N(0,1)$ indépendantes. Leur densité jointe sur $\RR^2$ est
\[
f (x, y ) = \frac{1}{2\pi} e^{-(x^2 +y^2 )/2}.
\]
Il est aisé de changer les coordonnées cartésiennes $(X,Y)$ par les coordonnées polaires $(R, \Theta)$:\\
\[
R^2 = X^2 + Y^2 ; Y = R \sin \Theta.
\]
La transformée inverse est donnée par
$X = R \cos\Theta$ et $Y = R \sin \Theta$.
La matrice jacobienne de la transformation $(r,\theta)$ vers $(x,y)$ est
\[
J = \begin{pmatrix} \cos \theta & -R \sin \theta \\ \sin \theta & R
  \cos \theta \end{pmatrix},
\]
de déterminant
$(\cos\theta)(r\cos\theta) - (\sin\theta)(-r\sin\theta) 
= r(\cos^2\theta+\sin^2\theta) = r$.

\end{frame}

\begin{frame}
\frametitle{Méthode de Box-Muller pour la loi normale}

Par conséquent, $(R,\Theta)$ a la densité
\[
t(r,\theta) = |J| f(x,y) =  r f(r \cos\theta,\, r \sin\theta) =
(r/2\pi) e^{-r^2/2}.
\]
Puisque cette densité ne dépend pas de $\theta$,
%  $f(x,y)$ depends only on $r^2 = x^2 + y^2$,
$\Theta$ doit avoir la densité uniforme sur $[0,2\pi]$.
En intégrant par rapport à $\theta$ sur l'intervalle $[0,2\pi]$,
nous obtenons que $R$ a la densité
\[
  f_R(r) = \int_0^{2\pi} t(r,\theta) d\theta = r e^{-r^2/2} \quad
  \mbox{pour } r\ge 0.
\]
La fonction de répartition correspondante est $F_R(r) = 1-e^{-r^2/2}$,
et sont inverse est donnée par
\[
  F_R^{-1}(U) = \sqrt{-2 \ln (1-U))}.
\]

\end{frame}

\begin{frame}
\frametitle{Méthode de Box-Muller pour la loi normale}

Il suffit alors de générer $\theta$ et $R$ indépendamment, en
utilisant $U_1$ et $U_2$, puis à transformer ces coordonnées polaires
en coordonnées rectangulaires $(X, Y )$, autrement dit
\begin{align*}
 X & = R \cos \theta = \cos(2 \pi U_1) \sqrt{ -2 \ln(U_2) } \\
 Y & = R \sin \theta = \sin(2 \pi U_1) \sqrt{ -2 \ln(U_2) }.
\end{align*}

\end{frame}

\begin{frame}[fragile]
\frametitle{Uniforme sur la sphère unité}

Nous souhaitons générer un vecteur $U$ uniformément distribué sur
$\mathcal{C}^k = \lbrace x \in \mathcal{R}^k\ |\ \|x\|_2 = 1 \rbrace$.
Ce genre de variables apparaît dans la génération de variables
multivariées elliptiques, utilisées notamment en gestions de risques.
La génération de fait en 3 lignes de codes!

\begin{lstlisting}
void generate_hypersphere
 (Random *ran, int k, double *x)
{
  int i;
  double norm;

  for (i = 0; i < k; i++)
    x[i] = ran_normal_icdf(ran, 0, 1);
  norm = cblas_dnrm2(n, x, 1);
  cblas_dscal(n, 1/norm, x, 1);
} 
\end{lstlisting}

\end{frame}

\begin{frame}[fragile]
\frametitle{Uniforme sur la sphère unité}

(Devroye, Chap. 5, Section 4)

Si $N_1,\ldots,N_k$ sont des normales i.i.d., alors, en notant $N = (
N_1, \ldots, N_k )$, $U = N/\|N\|_2$ est uniformément distribué
sur $\mathcal{C}^k$.

\mbox{}

Si $k = 2$, nous pouvons calculer un tel point en définissant une variable aléatoire $\Theta$ uniformément distribuée sur $[0,2\pi)$.

\mbox{}

Toutefois, le raisonnement ne peut pas s'étendre pour $k \geq 3$.

\mbox{}

Considérons $k = 3$. Nous serions alors tentés d'utiliser deux variables aléatoires uniformément distributées sur $[0, 2\pi)$.
Notons ces variables $\Theta$ et $B$, et supposons les indépendantes.

\end{frame}

\begin{frame}[fragile]
\frametitle{Uniforme sur la sphère unité}

Nous pouvons faire correspondre à chaque intervalle de $[0,2\pi)$ une probabilité, dont la valeur est fixée par la différence entre les extrémités de l'intervalle.
Considérons la surface décrite en laissant varier la première variable aléatoire $\Theta$ de $\theta_1$ à $\theta_2$,
et la seconde, $B$, de $\beta_1$ à $\beta_2$.
La surface vaut, en prenant $R$ comme rayon de la sphère,
\begin{align*}
S &= \left| \int_{\beta_1}^{\beta_2} \int_{\alpha_1}^{\alpha_2} R^2\cos \beta d\alpha d\beta \right|
\\
&= \left| R^2 (\alpha_2-\alpha_1)(sin \beta_1 - \sin \beta_2) \right|.
\end{align*}

\mbox{}

Par analogie avec la cartographie, nous dénommerons $\alpha$ la longitude et $\beta$ la latitude.

\end{frame}

\begin{frame}[fragile]
\frametitle{Uniforme sur la sphère unité}

Si la longueur de l'arc décrit en fonction de la latitude ne dépend que de la différence entre les latitudes extrême,
il n'en va pas de même pour la longitude: l'arc décrit est d'autant plus petit que la latitude est importante.

\mbox{}

Si les points étaient uniformément distribués sur la sphère, à deux surfaces de même valeur devraient correspondre à des probabilités identiques.
Nous pouvons constater que suivant la latitude, nous devrons varier la différence dans les longitudes.


\mbox{}

De manière équivalente, nous pouvons dire que les surfaces ne dépendant pas uniquement des différences d'angles.
Par conséquent, les points ne sont pas distribués uniformément sur la sphère.

\end{frame}

\end{document}
